% ==================================================================
% RESUMO
% ==================================================================

\chapter*{RESUMO}
\addcontentsline{toc}{chapter}{RESUMO}

\vspace{1cm}

Este trabalho tem como objetivo comparar o desempenho de três métodos de alocação de carteiras --- Markowitz, Equal Weight e Risk Parity --- utilizando dados de ativos da B3 no período de 2018 a 2019. Para a avaliação das carteiras, foram empregados o Índice de Sharpe, que mede o retorno ajustado ao risco total, e o Sortino Ratio, que considera apenas a volatilidade negativa, focando nos riscos de perda. O estudo adota uma abordagem quantitativa, descritiva e comparativa, utilizando ferramentas computacionais para otimização e análise. Os resultados pretendem oferecer insights relevantes para investidores em contextos de elevada volatilidade e incerteza, como o mercado brasileiro.

\vspace{0.5cm}

\noindent
\textbf{Palavras-chave:} Alocação de Carteiras; Markowitz; Equal Weight; Risk Parity; Índice de Sharpe; Sortino Ratio.