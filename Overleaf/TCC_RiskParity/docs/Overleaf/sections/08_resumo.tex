% ==================================================================
% RESUMO
% ==================================================================

\chapter*{RESUMO}
\addcontentsline{toc}{chapter}{RESUMO}

\vspace{1cm}

Este trabalho teve como objetivo comparar o desempenho de três estratégias de alocação de carteiras --- Otimização Média-Variância (Markowitz), Peso Igual (\textit{Equal Weight}) e Paridade de Risco (\textit{Risk Parity}) --- no mercado acionário brasileiro. \textbf{Metodologia:} O estudo adotou abordagem quantitativa out-of-sample com dados de 10 ações da B3 selecionadas cientificamente através de critérios objetivos de liquidez e performance no período 2014-2017. A análise utilizou janela de estimação (2016-2017) para calibração de parâmetros e janela de teste (2018-2019) para avaliação de performance, com rebalanceamento semestral das carteiras. \textbf{Resultados:} A análise empírica revelou superioridade da estratégia Markowitz (Sharpe Ratio: 1,86) comparada ao Risk Parity (1,21) e Equal Weight (1,20), embora com significância estatística marginal após correção de Bonferroni. \textbf{Conclusões:} Os resultados sugerem que a qualidade da seleção científica de ativos pode ser tão relevante quanto a sofisticação da técnica de otimização, oferecendo insights importantes para gestores em contextos de alta volatilidade como o mercado brasileiro.

\vspace{0.5cm}

\noindent
\textbf{Palavras-chave:} Alocação de Carteiras; Markowitz; Equal Weight; Risk Parity; Índice de Sharpe; Sortino Ratio.