% ==================================================================
% ABSTRACT
% ==================================================================

\chapter*{ABSTRACT}
\addcontentsline{toc}{chapter}{ABSTRACT}

\vspace{1cm}

This study aimed to compare the performance of three portfolio allocation strategies --- Mean-Variance Optimization (Markowitz), Equal Weight, and Risk Parity --- in the Brazilian stock market. \textbf{Methodology:} The study adopted a quantitative out-of-sample approach using data from 10 B3 stocks scientifically selected through objective liquidity and performance criteria during 2014-2017. The analysis used an estimation window (2016-2017) for parameter calibration and a test window (2018-2019) for performance evaluation, with semi-annual portfolio rebalancing. \textbf{Results:} The empirical analysis revealed superiority of the Markowitz strategy (Sharpe Ratio: 1.86) compared to Risk Parity (1.21) and Equal Weight (1.20), although with marginal statistical significance after Bonferroni correction. \textbf{Conclusions:} The results suggest that the quality of scientific asset selection may be as relevant as the sophistication of optimization techniques, offering important insights for managers in high volatility contexts such as the Brazilian market.

\vspace{0.5cm}

\noindent
\textbf{Keywords:} Portfolio Allocation; Markowitz; Equal Weight; Risk Parity; Sharpe Ratio; Sortino Ratio.