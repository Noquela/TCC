% ==================================================================
% 1 INTRODUÇÃO
% ==================================================================

\chapter{INTRODUÇÃO}

\section{PROBLEMA DE PESQUISA}

A alocação de ativos é amplamente reconhecida como um dos principais determinantes do desempenho de carteiras de investimento. Estudos clássicos, como o de Brinson, Hood e Beebower (1986), indicam que mais de 90\% da variância do retorno de uma carteira pode ser explicada por decisões de alocação estratégica de ativos, superando o impacto da seleção individual de ativos ou do timing de mercado.

No contexto brasileiro, essa decisão torna-se ainda mais crítica devido às características específicas do mercado emergente, incluindo maior volatilidade, sensibilidade a eventos políticos e correlações instáveis entre ativos. Embora existam diversas metodologias consolidadas internacionalmente --- como o modelo de Markowitz (1952), a estratégia Equal Weight e a abordagem Risk Parity --- sua eficácia relativa em mercados emergentes durante períodos de alta instabilidade permanece uma questão em aberto.

Especificamente no Brasil, o período de 2018-2019 apresentou características únicas de volatilidade extrema (superior a 25\% ao ano no Ibovespa) devido às incertezas eleitorais e mudanças econômicas estruturais. Esta conjuntura oferece um laboratório natural para testar a robustez e eficiência das diferentes estratégias de alocação, preenchendo uma lacuna específica na literatura acadêmica brasileira.

Em ambientes caracterizados por elevada volatilidade e incerteza, como frequentemente ocorre em mercados emergentes, a definição de uma estratégia de alocação eficiente torna-se ainda mais desafiadora, exigindo metodologias que consigam lidar com instabilidade, correlações variáveis e estimativas imperfeitas de risco e retorno (ILMANEN, 2022).

Entre as metodologias mais conhecidas e aplicadas na literatura acadêmica e no mercado estão o modelo de Média-Variância, proposto por Markowitz, a estratégia de alocação por pesos iguais (Equal Weight) e a metodologia de paridade de risco (Risk Parity). Cada uma dessas abordagens apresenta características específicas, vantagens próprias e limitações que precisam ser cuidadosamente analisadas em ambientes voláteis.

O modelo de Markowitz (1952) revolucionou a teoria financeira ao formalizar matematicamente a construção de carteiras eficientes, baseando-se na relação entre risco e retorno esperado. Seu principal objetivo é identificar a combinação ótima de ativos que maximize o retorno esperado para um nível específico de risco ou minimize o risco para determinado nível de retorno. Entretanto, esse modelo assume condições como a normalidade dos retornos dos ativos e a estabilidade das estimativas utilizadas, premissas que nem sempre se verificam na prática, especialmente em períodos de alta volatilidade ou crises financeiras.

Como alternativa de implementação mais simples, a estratégia Equal Weight distribui o capital igualmente entre todos os ativos selecionados na carteira, sem a necessidade de previsões complexas. Essa abordagem demonstra, em muitos estudos, ser bastante robusta em cenários de alta incerteza, apresentando desempenho comparável, ou até superior, a estratégias de otimização mais sofisticadas, especialmente em análises fora da amostra (DE MIGUEL; GARLAPPI; UPPAL, 2009). Por outro lado, sua simplicidade implica limitações, pois ignora características fundamentais dos ativos, como volatilidade e correlação, o que pode levar a concentrações de risco inadvertidas.

A metodologia de Risk Parity, por sua vez, busca uma distribuição mais equilibrada do risco total da carteira, atribuindo menores pesos a ativos mais voláteis e maiores pesos a ativos menos voláteis. Tal abordagem vem ganhando destaque nos últimos anos por produzir carteiras mais estáveis e menos suscetíveis a erros de estimativa, com desempenho sólido em diferentes cenários econômicos (MAILLARD; RONCALLI; TEILETCHE, 2010).

No cenário brasileiro, o período compreendido entre 2016 e 2019 foi marcado por alta volatilidade no mercado acionário, com o desvio-padrão anualizado dos retornos do Ibovespa oscilando entre 20% e 25%. Particularmente, os anos de 2018 e 2019 coincidiram com um contexto de incerteza política e financeira, principalmente em função das eleições presidenciais e das alterações no ambiente econômico subsequente. A literatura especializada demonstra que choques políticos influenciam diretamente os retornos de ações em mercados emergentes, especialmente de empresas com vínculos governamentais, e que eleições tendem a aumentar significativamente a volatilidade dos ativos no curto prazo (CARNAHAN; SAIEGH, 2020).

Diante desse contexto de instabilidade e alta incerteza, emerge a seguinte pergunta de pesquisa: \textbf{Qual das três estratégias de alocação de carteira (Markowitz, Equal Weight ou Risk Parity) apresenta melhor desempenho ajustado ao risco no mercado brasileiro durante períodos de alta volatilidade, utilizando metodologia out-of-sample sem look-ahead bias com dados de estimação de 2016-2017 aplicados ao período de teste de 2018-2019?}

Para responder a essa questão, o presente trabalho propõe uma análise comparativa entre as três estratégias mencionadas, utilizando dados de ativos negociados na B3 no período especificado. A comparação do desempenho será realizada com base em duas métricas amplamente reconhecidas na literatura financeira: o Índice de Sharpe, que avalia o retorno ajustado ao risco total da carteira, e o Sortino Ratio, que considera apenas os riscos de perdas.

Com essa abordagem, pretende-se contribuir para a identificação de estratégias de alocação mais eficientes no contexto brasileiro, gerando insights relevantes tanto para investidores quanto para gestores de recursos que buscam maximizar o retorno ajustado ao risco em ambientes de elevada volatilidade e imprevisibilidade.

\section{OBJETIVO GERAL}

Analisar comparativamente o desempenho das estratégias de alocação de carteira Markowitz, Equal Weight e Risk Parity no mercado brasileiro, utilizando metodologia out-of-sample rigorosa com dados de estimação de 2016-2017 e período de teste de 2018-2019, com base nos indicadores Índice de Sharpe e Sortino Ratio, a fim de identificar a estratégia mais eficiente em termos de retorno ajustado ao risco sem look-ahead bias.

\section{OBJETIVOS ESPECÍFICOS}

\begin{itemize}
    \item Selecionar uma amostra de 10 ações da B3, considerando critérios de liquidez, representatividade setorial e capitalização de mercado, com base na base de dados Economática.
    
    \item Calcular os retornos históricos dos ativos selecionados, estimar parâmetros como médias, volatilidades e covariâncias dos retornos.
    
    \item Implementar as três estratégias de alocação (Markowitz, Equal Weight e Risk Parity), programaticamente, por meio de ferramentas computacionais.
    
    \item Realizar o rebalanceamento semestral das carteiras durante o período de 2018 a 2019.
    
    \item Calcular os Índices de Sharpe e Sortino para cada carteira e para o período consolidado.
    
    \item Comparar os desempenhos obtidos, avaliando a eficiência de cada estratégia em ambientes de alta volatilidade e instabilidade política.
\end{itemize}

\section{JUSTIFICATIVA}

\subsection{Relevância Acadêmica}

A literatura internacional sobre estratégias de alocação de carteiras concentra-se predominantemente em mercados desenvolvidos, com poucos estudos específicos para mercados emergentes durante períodos de extrema volatilidade. Esta pesquisa contribui para preencher essa lacuna, oferecendo evidências empíricas sobre a eficácia comparativa das três principais metodologias de alocação no contexto brasileiro.

\subsection{Relevância Prática}

Os resultados obtidos podem orientar decisões práticas de gestores de recursos, investidores institucionais e individuais que operam no mercado brasileiro. A identificação da estratégia mais eficiente em ambientes de alta volatilidade pode resultar em melhores retornos ajustados ao risco, beneficiando diretamente os participantes do mercado.

\subsection{Originalidade}

A combinação específica do período analisado (2018-2019), do mercado estudado (B3) e das métricas utilizadas (Sharpe e Sortino) representa uma contribuição original à literatura acadêmica, especialmente considerando a raridade de estudos comparativos dessas três estratégias no contexto brasileiro.