% ==================================================================
% 1 INTRODUÇÃO
% ==================================================================

% A partir da introdução, mostra a numeração (continua a contagem)
\pagestyle{fancy}

\chapter{INTRODUÇÃO}

A alocação de ativos constitui uma das decisões centrais na gestão de carteiras de investimento, determinando como os recursos financeiros são distribuídos entre diferentes classes de ativos e, dentro de cada classe, entre ativos específicos. Esta decisão impacta diretamente o retorno esperado e o risco de uma carteira. Markowitz (1952), em seu trabalho pioneiro "Portfolio Selection" publicado no Journal of Finance, estabeleceu os fundamentos matemáticos para a construção de carteiras eficientes, introduzindo o conceito de diversificação baseada na correlação entre ativos. O trabalho de Markowitz demonstrou que o risco de uma carteira não é simplesmente a média dos riscos individuais dos ativos, mas depende fundamentalmente das correlações entre eles.

A relevância prática da alocação de ativos foi empiricamente demonstrada por Brinson, Hood e Beebower (1986), que analisaram carteiras institucionais americanas e concluíram que mais de 90\% da variabilidade dos retornos é explicada pela política de alocação estratégica, superando significativamente o impacto da seleção individual de títulos ou das decisões de entrada e saída do mercado. Este resultado, posteriormente confirmado por outros estudos, estabeleceu a alocação de ativos como fator determinante da performance de investimentos.

Desde o trabalho seminal de Markowitz, diferentes estratégias de alocação foram desenvolvidas. Além do modelo tradicional de otimização média-variância, duas abordagens alternativas ganharam destaque na literatura acadêmica e na prática de mercado: a estratégia de pesos iguais (equal weight) e a estratégia de paridade de risco (risk parity). A estratégia de pesos iguais consiste em alocar o mesmo percentual do capital para cada ativo da carteira. Embora aparentemente simples, demonstrou em diversos estudos uma performance surpreendentemente competitiva em relação a métodos mais sofisticados (DEMIGUEL; GARLAPPI; UPPAL, 2009). A estratégia de paridade de risco busca equalizar a contribuição de risco de cada ativo na carteira, ou seja, cada ativo contribui com a mesma quantidade de risco para o risco total da carteira, representando uma alternativa à diversificação tradicional baseada em valores monetários (MAILLARD; RONCALLI; TEILETCHE, 2010).

O mercado acionário brasileiro apresenta características específicas que podem influenciar a eficácia relativa dessas estratégias. Como mercado emergente, o Brasil exibe maior volatilidade, correlações instáveis entre ativos e sensibilidade elevada a fatores macroeconômicos específicos, incluindo taxa de câmbio, política monetária e preços de commodities. Adicionalmente, o mercado brasileiro caracteriza-se por concentração setorial significativa, com poucos setores representando a maior parte da capitalização total da bolsa de valores.

O período entre 2018 e 2019 oferece um contexto particularmente interessante para análise de estratégias de alocação no mercado brasileiro. Este período foi marcado por eventos que amplificaram a volatilidade e incerteza: processo eleitoral presidencial com alta polarização política, greve dos caminhoneiros em maio de 2018, incertezas sobre reformas estruturais e volatilidade nos preços de commodities. Durante este período, o índice Ibovespa apresentou volatilidade anualizada média superior à histórica, com episódios de volatilidade extrema que testaram a robustez das diferentes estratégias de alocação.

Apesar da importância do tema, a literatura acadêmica brasileira sobre estratégias de alocação de ativos permanece limitada. A maioria dos estudos comparativos entre estratégias de alocação concentra-se em mercados desenvolvidos, deixando uma lacuna no entendimento de como essas estratégias performam em mercados emergentes durante períodos de alta volatilidade. Esta lacuna é particularmente relevante considerando que as características específicas dos mercados emergentes podem alterar significativamente a eficácia relativa das diferentes abordagens.

Este trabalho tem como objetivo comparar três estratégias fundamentais de alocação de ativos no mercado acionário brasileiro: otimização média-variância de Markowitz, pesos iguais e paridade de risco. A análise será conduzida utilizando dados do período 2018-2019, aplicando metodologia que separa dados de estimação (2016-2017) dos dados de teste, evitando assim o uso de informações futuras na construção das carteiras. Os resultados contribuirão para o entendimento de qual estratégia apresenta melhor desempenho em mercados emergentes durante períodos de volatilidade elevada, oferecendo insights relevantes tanto para a literatura acadêmica quanto para a prática de gestão de recursos no Brasil.

\textbf{Questão de Pesquisa:}

\textbf{Qual das três estratégias de alocação (otimização de Markowitz, pesos iguais e paridade de risco) apresenta melhor desempenho ajustado ao risco no mercado acionário brasileiro durante período de alta volatilidade?}

Esta questão é relevante pois a literatura acadêmica sobre estratégias de alocação concentra-se predominantemente em mercados desenvolvidos, deixando uma lacuna no entendimento de como essas estratégias funcionam em mercados emergentes durante períodos de instabilidade.

\section{OBJETIVO GERAL}

Comparar o desempenho de três estratégias de alocação de ativos no mercado acionário brasileiro durante o período 2018-2019: otimização de Markowitz, pesos iguais e paridade de risco. A análise utilizará dados de 2016-2017 para construir as carteiras e avaliará seu desempenho no período 2018-2019, identificando qual estratégia apresenta melhor relação risco-retorno durante período de alta volatilidade.

\section{OBJETIVOS ESPECÍFICOS}

\begin{itemize}
    \item Selecionar ações brasileiras com base em critérios de liquidez e qualidade dos dados para formar o universo de investimento, utilizando informações disponíveis até 2017.

    \item Implementar as três estratégias de alocação: otimização de Markowitz, pesos iguais e paridade de risco.

    \item Aplicar metodologia que separa dados de construção (2016-2017) dos dados de teste (2018-2019), evitando o uso de informações futuras.

    \item Calcular métricas de desempenho incluindo retorno, volatilidade, índice de Sharpe e máximo rebaixamento (drawdown) para cada estratégia.

    \item Analisar como os eventos do período 2018-2019 (eleições, greve dos caminhoneiros) afetaram o desempenho das diferentes estratégias.

    \item Verificar se as diferenças de desempenho entre as estratégias são estatisticamente significativas.

    \item Discutir as implicações práticas dos resultados para gestores de recursos no mercado brasileiro.
\end{itemize}

\section{JUSTIFICATIVA}

Este trabalho é relevante por três razões principais. Primeiro, a literatura acadêmica sobre estratégias de alocação de ativos concentra-se principalmente em mercados desenvolvidos, havendo poucos estudos sobre o mercado brasileiro. Esta lacuna é importante pois mercados emergentes apresentam características distintas (maior volatilidade, correlações instáveis) que podem afetar a eficácia das diferentes estratégias.

Segundo, o período 2018-2019 oferece um contexto único para análise, combinando alta volatilidade política (eleições presidenciais) e econômica (greve dos caminhoneiros, incertezas sobre reformas). Compreender como diferentes estratégias se comportam durante períodos de instabilidade é fundamental para gestores de recursos que operam em mercados emergentes.

Terceiro, o mercado brasileiro de gestão de recursos movimenta trilhões de reais, tornando relevante qualquer melhoria nas estratégias de alocação. Os resultados podem orientar decisões práticas de gestores de fundos, fundos de pensão e investidores institucionais, contribuindo para melhor gestão de risco em carteiras domésticas.