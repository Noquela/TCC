% ==================================================================
% 2 REFERENCIAL TEÓRICO
% ==================================================================

\chapter{REFERENCIAL TEÓRICO}

\section{TEORIA DE PORTFÓLIO DE MARKOWITZ}

A moderna teoria de portfólio teve início com o trabalho pioneiro de Harry Markowitz (1952) publicado no Journal of Finance. Markowitz (1952) estabeleceu pela primeira vez uma base matemática para a construção de carteiras de investimento, demonstrando que o risco de uma carteira não é simplesmente a média dos riscos individuais dos ativos, mas depende das correlações entre eles.

O principal conceito introduzido por Markowitz (1952) é que investidores racionais buscam maximizar o retorno esperado para um dado nível de risco, ou minimizar o risco para um dado retorno esperado. Esta relação define a fronteira eficiente, que representa o conjunto de carteiras ótimas disponíveis aos investidores.

Segundo Markowitz (1952), o risco de uma carteira pode ser calculado pela seguinte fórmula simplificada: quando dois ativos possuem correlação perfeita positiva (+1), o risco da carteira é a média ponderada dos riscos individuais. Quando a correlação é menor que +1, o risco da carteira será menor que essa média, demonstrando o benefício da diversificação.

A teoria de Markowitz (1952) assume que os investidores são aversos ao risco e que os retornos dos ativos seguem distribuição normal. Embora essas premissas tenham sido questionadas por estudos posteriores, o framework continua sendo a base fundamental para estratégias modernas de alocação de ativos. A relevância prática desta teoria foi empiricamente demonstrada por Brinson, Hood e Beebower (1986), que analisaram carteiras institucionais americanas e concluíram que mais de 90\% da variabilidade dos retornos é explicada pela política de alocação estratégica.

A implementação prática da teoria de Markowitz envolve a maximização do índice de Sharpe, desenvolvido por Sharpe (1964) no contexto do Capital Asset Pricing Model. O índice de Sharpe mede o retorno em excesso por unidade de risco e é calculado como:

\begin{equation}
\text{Sharpe} = \frac{R_p - R_f}{\sigma_p}
\end{equation}

onde $R_p$ é o retorno da carteira, $R_f$ é a taxa livre de risco e $\sigma_p$ é a volatilidade da carteira. Segundo Sharpe (1964), esta métrica é amplamente utilizada para comparar estratégias de investimento pois considera tanto retorno quanto risco.

A otimização de Markowitz apresenta limitações práticas importantes. Michaud (1989) identificou o "enigma da otimização", demonstrando que carteiras teoricamente ótimas frequentemente apresentam desempenho decepcionante fora da amostra devido à instabilidade das estimativas paramétricas. Chopra e Ziemba (1993) quantificaram esta sensibilidade, demonstrando que erros nas estimativas de retorno esperado têm impacto na performance da carteira 11 vezes maior que erros equivalentes nas estimativas de variância. Esta descoberta sugere que a qualidade das estimativas de retorno esperado é crítica para o sucesso da implementação da estratégia de Markowitz.

\section{ESTRATÉGIA DE PESOS IGUAIS}

A estratégia de pesos iguais consiste em alocar o mesmo percentual do capital para cada ativo da carteira. Para uma carteira com $n$ ativos, cada ativo recebe peso $w_i = 1/n$. DeMiguel, Garlappi e Uppal (2009) demonstraram que esta estratégia simples frequentemente supera métodos de otimização sofisticados quando aplicada fora da amostra.

Os autores compararam 14 estratégias de alocação diferentes e concluíram que nenhuma superou consistentemente a estratégia 1/N em termos de índice de Sharpe, retorno ajustado pela utilidade ou rotatividade. Este resultado surpreendente ocorre porque os erros de estimação nas estratégias otimizadas superam os benefícios da otimização (DEMIGUEL; GARLAPPI; UPPAL, 2009).

Segundo DeMiguel, Garlappi e Uppal (2009), a janela de estimação necessária para que estratégias baseadas em média-variância superem a estratégia 1/N é de aproximadamente 3000 meses para carteiras com 25 ativos e 6000 meses para carteiras com 50 ativos. Como essas janelas são impraticáveis, a estratégia de pesos iguais torna-se uma alternativa robusta para investidores.

\section{ESTRATÉGIA DE PARIDADE DE RISCO}

A estratégia de paridade de risco, também conhecida como Equal Risk Contribution (ERC), busca equalizar a contribuição de risco de cada ativo para o risco total da carteira. Maillard, Roncalli e Teiletche (2010) formalizaram esta abordagem, que representa uma alternativa à diversificação tradicional baseada em valores monetários.

Na paridade de risco, o objetivo é que cada ativo contribua com a mesma quantidade de risco para a volatilidade total da carteira. A contribuição de risco do ativo $i$ pode ser expressa como o produto entre o peso do ativo e sua sensibilidade marginal ao risco da carteira. Para que todos os ativos tenham contribuição igual, esta deve ser $1/n$ do risco total (MAILLARD; RONCALLI; TEILETCHE, 2010).

A estratégia de paridade de risco oferece diversificação de risco mais efetiva que a diversificação por capital, especialmente quando os ativos possuem volatilidades muito diferentes. Segundo Maillard, Roncalli e Teiletche (2010), esta abordagem maximiza a diversificação ex-ante sem depender de estimativas de retornos esperados, tornando-a mais robusta que estratégias baseadas em otimização média-variância.

A implementação prática da paridade de risco requer algoritmos iterativos, pois não existe solução analítica fechada. O algoritmo mais comum utiliza o método do gradiente, onde os pesos são ajustados iterativamente na direção que minimiza a diferença entre as contribuições de risco. O processo continua até que a diferença entre as contribuições de risco de todos os ativos seja menor que uma tolerância predefinida, tipicamente 1e-6.

A contribuição de risco marginal do ativo $i$ pode ser calculada como:

\begin{equation}
\text{Contribuição}_i = w_i \times \frac{\partial \sigma_p}{\partial w_i}
\end{equation}

onde $\sigma_p$ é a volatilidade da carteira. Para carteiras com paridade de risco perfeita, todas as contribuições devem ser iguais a $\sigma_p / n$, onde $n$ é o número de ativos.

\section{MERCADOS EMERGENTES E CARACTERÍSTICAS ESPECÍFICAS}

Os mercados emergentes apresentam características distintas dos mercados desenvolvidos que podem afetar significativamente a eficácia das diferentes estratégias de alocação. Harvey (1995) identificou propriedades específicas destes mercados que desafiam as premissas tradicionais da teoria de portfólio: maior volatilidade dos retornos, correlações instáveis entre ativos e maior sensibilidade a choques políticos e econômicos locais.

O mercado brasileiro, como mercado emergente, caracteriza-se por concentração setorial significativa e sensibilidade elevada a variáveis macroeconômicas específicas, incluindo taxa de câmbio, política monetária e preços de commodities. Harvey (1995) demonstra que esta concentração setorial pode limitar os benefícios da diversificação tradicional durante períodos de estresse do mercado.

O período entre 2018 e 2019 no Brasil oferece um contexto particularmente interessante para análise de estratégias de alocação. Este período foi marcado por eventos que amplificaram a volatilidade: processo eleitoral presidencial com alta polarização política em 2018, greve dos caminhoneiros em maio de 2018 que paralisou a economia por dez dias, incertezas sobre reformas estruturais e volatilidade nos preços de commodities. Durante este período, o índice Ibovespa apresentou volatilidade anualizada média superior à histórica, confirmando as características de instabilidade típicas de mercados emergentes identificadas por Harvey (1995).

\section{CRITÉRIOS DE SELEÇÃO DE ATIVOS}

A seleção de ativos constitui etapa fundamental que antecede a aplicação das estratégias de alocação. Markowitz (1959) reconheceu que "a escolha dos títulos a serem incluídos no portfólio é tão importante quanto a determinação de suas proporções ótimas". Black e Litterman (1992) reforçaram esta perspectiva demonstrando que a qualidade do universo inicial influencia dramaticamente os resultados de otimização.

\subsection{Critérios de Liquidez}

A importância da liquidez na seleção de ativos foi estabelecida por Amihud (2002), que desenvolveu medidas de iliquidez baseadas na relação entre retorno absoluto e volume de negociação. Roll (1984) propôs métricas operacionais para avaliar liquidez, incluindo dias com retorno zero e estimativas de bid-ask spread.

Este trabalho utiliza critérios rigorosos de liquidez: volume médio diário mínimo de R\$ 5 milhões, presença em bolsa superior a 80\% dos dias úteis, e menos de 20\% de dias com retorno zero. Estes filtros garantem que os ativos selecionados sejam efetivamente negociáveis durante todo o período de análise.

\subsection{Score Composto de Seleção}

Para integrar múltiplos critérios de qualidade, este trabalho desenvolve um score composto baseado em quatro dimensões fundamentais: momentum, volatilidade, máximo rebaixamento e desvio negativo. Os pesos utilizados são baseados na literatura acadêmica: 35\% para momentum (JEGADEESH; TITMAN, 1993), 25\% para volatilidade, e 20\% para cada métrica de risco extremo.

O momentum 12-1, amplamente documentado na literatura como fator preditivo robusto (JEGADEESH; TITMAN, 1993), mede a performance acumulada nos 12 meses anteriores excluindo o último mês para evitar efeitos de reversão de curto prazo. A volatilidade anualizada captura a estabilidade histórica do ativo, enquanto máximo rebaixamento e desvio negativo medem exposição a riscos extremos.

O score final é calculado como:

\begin{align}
\text{Score} &= 0,35 \times \text{Rank}_{\text{momentum}} + 0,25 \times (1 - \text{Rank}_{\text{vol}}) \nonumber \\
&\quad + 0,20 \times \text{Rank}_{\text{MDD}} + 0,20 \times (1 - \text{Rank}_{\text{downside}})
\end{align}

onde os ranks são percentuais (0 a 1) e as métricas de risco são invertidas para que menores valores representem melhor qualidade.

\subsection{Dimensão da Carteira e Diversificação}

A escolha de 10 ativos para as carteiras baseia-se em evidências empíricas sobre diversificação ótima. Evans e Archer (1968) demonstraram que a maior parte dos benefícios de diversificação é obtida com 8 a 16 ações. Statman (1987) confirmou que carteiras com 10 a 15 ações bem selecionadas capturam aproximadamente 95\% dos benefícios de diversificação disponíveis.

O uso de 10 ativos representa compromisso entre diversificação adequada e praticabilidade de gestão. DeMiguel, Garlappi e Uppal (2009) utilizaram carteiras de tamanhos similares em seu estudo seminal, demonstrando que este tamanho é apropriado para comparações entre estratégias de alocação. Carteiras menores sofreriam de subdiversificação, enquanto carteiras muito maiores diluiriam os efeitos das diferentes estratégias de alocação.

\subsection{Frequência de Rebalanceamento}

O rebalanceamento semestral adotado neste trabalho reflete equilíbrio entre capturar oportunidades de realocação e controlar custos de transação. Constantinides (1986) demonstrou que rebalanceamentos muito frequentes podem ser prejudiciais devido aos custos de transação, enquanto rebalanceamentos muito espaçados permitem que as carteiras se desviem significativamente das alocações ótimas.

A literatura documenta que o rebalanceamento semestral é uma frequência robusta para comparações entre estratégias. Brinson, Hood e Beebower (1986) utilizaram rebalanceamentos trimestrais em seu estudo clássico, enquanto estudos mais recentes como DeMiguel, Garlappi e Uppal (2009) adotaram frequências mensais, trimestrais e anuais. A escolha semestral situa-se no meio deste espectro, sendo suficiente para capturar mudanças nas condições de mercado.

\section{MÉTRICAS DE AVALIAÇÃO DE PERFORMANCE}

\subsection{Volatilidade e Medidas de Risco}

A volatilidade constitui medida fundamental de risco em finanças. Para dados de retornos mensais, a volatilidade anualizada é calculada como:

\begin{equation}
\sigma_{\text{anual}} = \sigma_{\text{mensal}} \times \sqrt{12}
\end{equation}

Esta conversão assume independência dos retornos mensais e aplica a propriedade de escalabilidade da variância para processos estocásticos, conforme estabelecido na literatura de séries temporais financeiras.

\subsection{Máximo Rebaixamento (Maximum Drawdown)}

O máximo rebaixamento mede a maior perda acumulada desde um pico anterior até o vale subsequente. Segundo Martin e McCann (1989), esta métrica é fundamental para avaliar o pior cenário experimentado pelos investidores. O cálculo é realizado como:

\begin{equation}
\text{MDD} = \max_{t \in [0,T]} \left[ \frac{\max_{s \in [0,t]} V_s - V_t}{\max_{s \in [0,t]} V_s} \right]
\end{equation}

onde $V_t$ é o valor acumulado da carteira no tempo $t$. Esta métrica oferece perspectiva única sobre tail risk e experiência real do investidor, sendo amplamente utilizada por investidores institucionais para estabelecer limites de risco.

\subsection{Índice de Sortino}

O índice de Sortino, desenvolvido por Sortino e Price (1994), refinam o conceito de risco ao considerar apenas volatilidade negativa. Este índice alinha-se melhor com preferências reais de investidores, que tipicamente se preocupam mais com perdas que com ganhos:

\begin{equation}
\text{Sortino} = \frac{R_p - \text{MAR}}{\sigma_{\text{downside}}}
\end{equation}

onde MAR é o retorno mínimo aceitável (normalmente a taxa livre de risco) e $\sigma_{\text{downside}}$ é o desvio padrão calculado apenas com retornos abaixo do MAR:

\begin{equation}
\sigma_{\text{downside}} = \sqrt{\frac{1}{T} \sum_{t=1}^{T} \min(R_t - \text{MAR}, 0)^2}
\end{equation}

\subsection{Metodologia Out-of-Sample}

Para evitar look-ahead bias, este trabalho implementa metodologia rigorosa out-of-sample conforme estabelecido por DeMiguel, Garlappi e Uppal (2009). Os dados são divididos em período de estimação (2016-2017) e período de teste (2018-2019), utilizando apenas informações do período de estimação para construção das carteiras.

Esta separação temporal garante que nenhuma informação futura seja utilizada na tomada de decisões de alocação, simulando condições realistas de investimento. O rebalanceamento é realizado semestralmente, balanceando custos de transação com necessidade de ajustes nas alocações.

\subsection{Testes de Significância Estatística}

Para verificar se as diferenças de performance entre estratégias são estatisticamente significativas, aplicam-se testes específicos para comparação de índices de Sharpe. O teste de Jobson e Korkie (1981), posteriormente corrigido por Memmel (2003), testa a hipótese nula de que dois índices de Sharpe são iguais:

\begin{equation}
t = \frac{\text{SR}_1 - \text{SR}_2}{\sqrt{\text{Var}(\text{SR}_1 - \text{SR}_2)}}
\end{equation}

onde a variância da diferença considera a correlação entre as estratégias e é ajustada para amostras finitas. Este teste permite conclusões estatisticamente robustas sobre a superioridade de uma estratégia em relação às outras.

\subsection{Taxa Livre de Risco}

A taxa livre de risco utilizada nos cálculos de índices de Sharpe e Sortino é a taxa SELIC, que representa o benchmark livre de risco no mercado brasileiro. A SELIC é amplamente aceita na literatura acadêmica brasileira como proxy adequada para a taxa livre de risco, sendo utilizada pelo Banco Central do Brasil como instrumento principal de política monetária.

A conversão da SELIC anual para frequência mensal é realizada através da fórmula de capitalização composta: $(1 + \text{SELIC}_{\text{anual}})^{1/12} - 1$. Esta abordagem preserva a equivalência financeira entre as taxas, sendo metodologicamente superior à simples divisão por 12 que assumiria capitalização linear.

\section{JUSTIFICATIVA DO PERÍODO DE ANÁLISE}

A escolha do período 2018-2019 para análise não é arbitrária, mas fundamentada em características específicas que tornam este período particularmente adequado para testar estratégias de alocação em condições de estresse. Schwert (1989) argumenta que períodos de alta volatilidade oferecem testes mais rigorosos para estratégias de investimento, pois amplificam as diferenças entre abordagens alternativas. Campbell et al. (2001) reforçam esta perspectiva demonstrando que choques econômicos revelam propriedades latentes das estratégias de investimento que não são observáveis em períodos de normalidade.

\subsection{Contexto Macroeconômico e Político}

O biênio 2018-2019 no Brasil foi caracterizado por múltiplos choques simultâneos que testaram a robustez das estratégias de alocação. O processo eleitoral presidencial de 2018 foi marcado por alta polarização política e incerteza sobre o futuro das políticas econômicas. Segundo dados do Banco Central do Brasil, o Índice de Incerteza da Política Econômica (EPU-Brazil) atingiu níveis históricos durante o período eleitoral, refletindo a ansiedade dos agentes econômicos.

A greve dos caminhoneiros em maio de 2018 constitui evento particularmente relevante para este estudo. Durante dez dias, a paralisação afetou diferentemente diversos setores da economia: empresas de bens de consumo enfrentaram rupturas nas cadeias de suprimento, enquanto empresas de energia e telecomunicações mostraram maior resiliência. Este choque idiossincrático oferece laboratório natural para observar como diferentes estratégias de alocação reagiram a impactos setoriais assimétricos.

\subsection{Volatilidade dos Mercados Financeiros}

Durante 2018-2019, o mercado acionário brasileiro experimentou volatilidade significativamente elevada. O índice Ibovespa apresentou volatilidade anualizada média de 28,5\% em 2018, comparado à média histórica de 23\% (B3, 2019). Esta elevação da volatilidade foi acompanhada por episódios de correlação aumentada entre ativos, fenômeno documentado por Longin e Solnik (2001) como característico de períodos de estresse.

A instabilidade cambial também caracterizou o período, com o real brasileiro depreciando aproximadamente 17\% em relação ao dólar americano em 2018. Esta volatilidade cambial teve impactos diferenciados nas empresas listadas: exportadoras de commodities beneficiaram-se da depreciação, enquanto empresas com elevada dívida em moeda estrangeira enfrentaram pressões adicionais.

\subsection{Choques Setoriais e Oportunidades de Diversificação}

A volatilidade nos preços de commodities durante 2018-2019 afetou especialmente empresas dos setores de mineração e petróleo, que representam parcela significativa do índice Ibovespa. O preço do petróleo Brent variou entre US\$ 50 e US\$ 85 por barril durante o período, enquanto o minério de ferro enfrentou volatilidade relacionada às políticas comerciais entre Estados Unidos e China.

Estes choques setoriais criaram oportunidades de diversificação que foram exploradas diferentemente pelas estratégias de alocação. A estratégia de pesos iguais, por sua natureza, manteve exposição constante a todos os setores. A paridade de risco ajustou automaticamente para reduzir exposição a setores mais voláteis, enquanto a otimização de Markowitz reagiu às mudanças nas correlações e volatilidades estimadas.

\subsection{Relevância para Testes de Estratégias de Alocação}

Este conjunto de fatores criou ambiente de teste natural para comparar estratégias de alocação, pois diferentes abordagens reagiram de forma distinta aos choques. Períodos de baixa volatilidade tendem a mascarar diferenças entre estratégias, enquanto períodos de estresse revelam suas características fundamentais de risco e retorno, conforme documentado por Pastor e Stambaugh (2002).

A literatura acadêmica sobre estratégias de alocação frequentemente utiliza períodos de crise para validar a robustez dos métodos propostos. DeMiguel, Garlappi e Uppal (2009) incluíram dados da crise das empresas pontocom e da crise financeira de 2008 em suas análises comparativas. Similarmente, este trabalho utiliza o período 2018-2019 como teste de estresse natural para as estratégias de alocação no contexto brasileiro.

\section{ANÁLISE DE CORRELAÇÃO E ESTRUTURA DE DEPENDÊNCIA}

A correlação entre ativos constitui elemento central na teoria de portfólio, determinando os benefícios potenciais da diversificação. Markowitz (1952) estabeleceu que correlações baixas ou negativas entre ativos permitem redução do risco da carteira sem sacrificar retorno esperado. A mensuração adequada da estrutura de correlação é, portanto, fundamental para a implementação efetiva das estratégias de alocação.

\subsection{Correlação Linear de Pearson}

A correlação linear de Pearson, utilizada tradicionalmente na literatura de portfólio, mede a intensidade da relação linear entre retornos de dois ativos. Para ativos $i$ e $j$, a correlação é calculada como:

\begin{equation}
\rho_{i,j} = \frac{\text{Cov}(R_i, R_j)}{\sigma_i \sigma_j}
\end{equation}

onde $\text{Cov}(R_i, R_j)$ é a covariância entre os retornos e $\sigma_i, \sigma_j$ são os desvios padrão individuais. Valores próximos a +1 indicam forte correlação positiva, valores próximos a -1 indicam forte correlação negativa, e valores próximos a zero indicam ausência de relação linear.

\subsection{Instabilidade das Correlações}

Longin e Solnik (2001) documentaram que correlações entre ativos financeiros são instáveis ao longo do tempo, aumentando significativamente durante períodos de estresse de mercado. Este fenômeno, conhecido como "correlation breakdown", reduz os benefícios da diversificação exatamente quando os investidores mais precisam dela.

No contexto de mercados emergentes, esta instabilidade é ainda mais pronunciada. Forbes e Rigobon (2002) demonstraram que correlações condicionais podem aumentar substancialmente durante crises, limitando a eficácia da diversificação internacional. Para estratégias de alocação, isto implica que correlações estimadas em períodos de normalidade podem subestimar o risco durante períodos de estresse.

\subsection{Correlação na Seleção de Ativos}

A análise da correlação média entre ativos pode ser utilizada como critério adicional na seleção do universo de investimento. Ativos com correlações excessivamente altas entre si oferecem benefícios limitados de diversificação, tornando preferível a seleção de ativos com correlações mais baixas com o restante da carteira.

Para implementação prática, pode-se calcular a correlação média de cada ativo candidato com os demais ativos já selecionados. Ativos com correlação média superior a um threshold (por exemplo, 0,8) podem ser excluídos para favorecer maior diversificação. Esta abordagem simples melhora a qualidade do universo de investimento sem adicionar complexidade desnecessária ao processo de seleção.