% ==================================================================
% 3 METODOLOGIA
% ==================================================================

\chapter{METODOLOGIA}

\section{TIPO DE PESQUISA E ESTRATÉGIA METODOLÓGICA}

Este estudo é de natureza quantitativa, descritiva e comparativa, com foco na avaliação do desempenho de diferentes estratégias de alocação de ativos financeiros. A pesquisa adota uma abordagem empírica, utilizando dados históricos do mercado financeiro brasileiro para a construção e análise das carteiras.

\section{PERÍODO E AMBIENTE DE ESTUDO}

O horizonte temporal da análise compreende o período de janeiro de 2018 a dezembro de 2019, um momento de alta volatilidade e instabilidade política no Brasil. O ambiente de estudo é a B3 -- Brasil Bolsa Balcão, principal bolsa de valores brasileira.

\section{SELEÇÃO DOS ATIVOS}

A seleção dos ativos da amostra segue critérios rigorosamente \textbf{ex-ante}, baseados exclusivamente em informações disponíveis antes do período de teste (2018-2019):

\begin{itemize}
    \item \textbf{Alta liquidez histórica:} Volume médio diário de negociação superior a R\$ 50 milhões, calculado exclusivamente com base no histórico de janeiro de 2016 a dezembro de 2017;
    
    \item \textbf{Diversificação setorial:} Inclusão de ações de diferentes setores da economia brasileira;
    
    \item \textbf{Capitalização de mercado:} Empresas de maior valor de mercado em dezembro de 2017, geralmente pertencentes ao índice Ibovespa na época.
\end{itemize}

\textbf{Importante:} Esta metodologia de seleção elimina o survivorship bias, pois todos os critérios são baseados em informações anteriores ao período de teste. A seleção final de 10 ativos foi baseada na aplicação manual dos critérios acima à base de dados histórica, priorizando empresas com alta liquidez, diversificação setorial e que atendiam aos requisitos de dados completos para o período 2016-2019. Não foi aplicado qualquer filtro baseado no desempenho durante 2018-2019, garantindo que a amostra reflita decisões que poderiam ter sido tomadas ex-ante.

\subsection{Critérios de Seleção Final}

\begin{itemize}
    \item \textbf{Representatividade de mercado:} Preferência por empresas pertencentes ao índice Ibovespa em janeiro de 2018, garantindo representatividade do mercado brasileiro;
    
    \item \textbf{Diversificação setorial:} Máximo de 2 ativos por setor econômico, baseada na classificação setorial da Economática, visando reduzir riscos de concentração setorial;
    
    \item \textbf{Capitalização de mercado:} Ranking por valor de mercado em janeiro de 2018, priorizando empresas de maior porte dentro de cada setor selecionado.
\end{itemize}

Estes critérios visam evitar viés de sobrevivência (survivorship bias) e garantir que a amostra seja representativa do universo de investimentos disponível para gestores profissionais no início do período de análise.

\subsection{Distribuição Setorial da Base de Dados}

A base de dados da Economática apresenta a seguinte distribuição setorial entre as 508 empresas analisadas:

\begin{itemize}
    \item \textbf{Outros:} 117 empresas (23,0\%)
    \item \textbf{Energia Elétrica:} 65 empresas (12,8\%)
    \item \textbf{Finanças e Seguros:} 45 empresas (8,9\%)
    \item \textbf{Comércio:} 40 empresas (7,9\%)
    \item \textbf{Construção:} 33 empresas (6,5\%)
    \item \textbf{Siderurgia \& Metalurgia:} 32 empresas (6,3\%)
    \item \textbf{Demais setores:} 176 empresas (34,6\%)
\end{itemize}

A partir desta base, foram selecionados 10 ativos representativos, garantindo diversificação setorial e alta liquidez, conforme os critérios estabelecidos.

\subsection{Caracterização da Amostra Final}

A seleção final resultou em uma carteira diversificada de 10 ativos representativos do mercado brasileiro:

\begin{table}[H]
\centering
\caption{Ativos Selecionados para Análise - Características e Performance}
\begin{tabular}{llcc}
\toprule
\textbf{Código} & \textbf{Empresa} & \textbf{Retorno} & \textbf{Volatilidade} \\
& & \textbf{Anual (\%)} & \textbf{Anual (\%)} \\
\midrule
PETR4 & Petróleo Brasileiro S.A. & 25,0 & 33,1 \\
VALE3 & Vale S.A. & 15,9 & 23,6 \\
ITUB4 & Itaú Unibanco Holding S.A. & 10,3 & 21,9 \\
BBDC4 & Banco Bradesco S.A. & 12,4 & 29,1 \\
ABEV3 & Ambev S.A. & -5,2 & 26,7 \\
B3SA3 & B3 S.A. & 27,5 & 28,3 \\
WEGE3 & WEG S.A. & 33,8 & 24,0 \\
RENT3 & Localiza Rent a Car S.A. & 33,9 & 27,9 \\
LREN3 & Lojas Renner S.A. & 25,8 & 23,3 \\
ELET3 & Centrais Elétricas Brasileiras & 16,5 & 62,2 \\
\bottomrule
\end{tabular}
\label{tab:ativos_selecionados}
\end{table}

\subsection{Justificativa da Seleção}

Esta composição garante:
\begin{itemize}
    \item \textbf{Diversificação setorial:} 7 setores distintos representados
    \item \textbf{Representatividade:} 46,6\% do peso total do Ibovespa em janeiro de 2018
    \item \textbf{Liquidez:} Volume médio diário superior a R\$ 100 milhões para todos os ativos
    \item \textbf{Capitalização:} Empresas de grande porte com histórico consolidado
    \item \textbf{Sobrevivência:} Todos os ativos mantiveram negociação ativa durante 2018-2019
\end{itemize}

\section{COLETA E TRATAMENTO DOS DADOS}

Os dados históricos de preços ajustados dos ativos foram coletados da base de dados Economática, abrangendo o período de 2014 a 2019. A base contém informações detalhadas de 508 empresas listadas na B3, incluindo:

\begin{itemize}
    \item \textbf{Dados de cotações:} preços de abertura, fechamento, máximo, mínimo e médio, todos ajustados por proventos (dividendos, juros sobre capital próprio, bonificações e desdobramentos);
    \item \textbf{Volume de negociação:} quantidade de negócios, títulos e volume financeiro;
    \item \textbf{Classificação setorial:} setor Economática, setor econômico Bovespa e segmento Bovespa;
    \item \textbf{Códigos dos ativos:} identificação única de cada papel negociado.
\end{itemize}

A utilização de preços ajustados por proventos é fundamental para evitar distorções na análise de retornos, uma vez que eventos corporativos como dividendos causam quedas artificiais nos preços das ações na data ex-dividendo. Sem esse ajuste, os retornos calculados seriam inconsistentes e não refletiriam a performance real dos investimentos.

O tratamento dos dados inclui:

\begin{itemize}
    \item Remoção de ativos com dados faltantes ou séries históricas incompletas no período de análise;
    \item Cálculo dos retornos mensais e anualizados com base nos preços de fechamento ajustados;
    \item Estimativa das volatilidades individuais e matriz de covariância entre os retornos dos ativos;
    \item Definição da taxa livre de risco como o CDI médio anualizado do período.
\end{itemize}

\section{CONSTRUÇÃO DAS CARTEIRAS}

Serão implementadas três estratégias de alocação:

\subsection{Markowitz (Média-Variância)}
Otimização para maximizar o Índice de Sharpe, com restrições de soma dos pesos igual a 1, ausência de vendas a descoberto, e peso mínimo de 0,1\% por ativo para garantir diversificação mínima.

\subsection{Equal Weight}
Alocação igualitária do capital entre os ativos.

\subsection{Risk Parity}
Alocação baseada na contribuição igual de risco de cada ativo, conforme definido na Equação \ref{eq:risk_parity} (seção de referencial teórico). Este trabalho implementa o modelo ERC (Equal Risk Contribution) que considera a matriz de covariância completa para equalizar as contribuições marginais de risco.

As carteiras serão construídas usando a linguagem Python, com bibliotecas como pandas, NumPy e scipy.optimize.

\subsection{Limites Práticos e Restrições de Otimização}

Para garantir implementabilidade prática das estratégias, foram aplicadas as seguintes restrições durante o processo de otimização:

\begin{itemize}
    \item \textbf{Proibição de vendas a descoberto:} $w_i \geq 0$ para todos os ativos $i$, eliminando estratégias que dependem de posições curtas
    
    \item \textbf{Peso mínimo por ativo:} $w_i \geq 0,1\%$ para garantir diversificação mínima e evitar alocações desprezíveis que aumentam custos operacionais
    
    \item \textbf{Restrição orçamentária:} $\sum_{i=1}^{n} w_i = 1$, assegurando investimento total do capital disponível
    
    \item \textbf{Limite setorial implícito:} Máximo de 3 ativos por setor econômico (aplicado na fase de seleção ex-ante), reduzindo concentração setorial
    
    \item \textbf{Ausência de limite individual superior:} Não foram impostos tetos por ativo, permitindo concentração natural conforme resultado da otimização matemática
    
    \item \textbf{Rebalanceamento semestral:} Frequência fixa para controlar custos de transação e evitar over-trading
\end{itemize}

Essas restrições refletem limitações práticas enfrentadas por gestores de recursos reais, equilibrando otimalidade teórica com viabilidade operacional.

\section{MÉTRICAS DE AVALIAÇÃO}

Para avaliação comparativa das estratégias, serão utilizadas as seguintes métricas de risco-retorno:

\subsection{Índice de Sharpe}
Mede o retorno excedente por unidade de risco total:
\begin{equation}
Sharpe = \frac{R_p - R_f}{\sigma_p}
\end{equation}
onde $R_p$ é o retorno do portfólio, $R_f$ é a taxa livre de risco e $\sigma_p$ é o desvio-padrão dos retornos.

\subsection{Sortino Ratio}
Considera apenas a volatilidade negativa (downside risk):
\begin{equation}
Sortino = \frac{R_p - R_f}{\sigma_{down}}
\end{equation}
onde $\sigma_{down}$ é o desvio-padrão dos retornos abaixo da taxa livre de risco.

\subsection{Information Ratio}
Mede o retorno ativo por unidade de tracking error:
\begin{equation}
IR = \frac{R_p - R_b}{TE}
\end{equation}
onde $R_b$ é o retorno do Ibovespa (B3 Oficial) e $TE$ é o tracking error (desvio-padrão dos retornos ativos).

A taxa livre de risco utilizada é de 6,24\% ao ano (0,52\% ao mês), baseada na média do CDI durante o período 2018-2019.

\section{METODOLOGIA OUT-OF-SAMPLE}

Para garantir rigor acadêmico e eliminar look-ahead bias, implementamos uma metodologia robusta de análise out-of-sample com rebalanceamento semestral e janela móvel de estimação.

\subsection{Estrutura Temporal}

O estudo abrange 24 observações mensais (2018-2019), divididas estrategicamente em:
\begin{itemize}
    \item \textbf{Período de Estimação:} 2016-2017 (24 observações iniciais)
    \item \textbf{Período de Teste:} 2018-2019 (24 observações mensais, jan/2018 a dez/2019)
\end{itemize}

\subsection{Períodos de Rebalanceamento}

\begin{table}[H]
\centering
\caption{Estrutura de Rebalanceamento Out-of-Sample}
\begin{tabular}{lccc}
\toprule
\textbf{Período} & \textbf{Janela Estimação} & \textbf{Período Teste} & \textbf{Meses Teste} \\
\midrule
1 & Jan/2016 - Jan/2018 & Jan/2018 - Jul/2018 & 6 meses \\
2 & Jul/2016 - Jul/2018 & Jul/2018 - Jan/2019 & 6 meses \\
3 & Jan/2017 - Jan/2019 & Jan/2019 - Jul/2019 & 6 meses \\
4 & Jul/2017 - Jul/2019 & Jul/2019 - Dez/2019 & 5 meses \\
\bottomrule
\end{tabular}
\label{tab:rebalanceamento}
\end{table}

\subsection{Eliminação do Look-Ahead Bias}

Esta metodologia garante que:
\begin{enumerate}
    \item Apenas informações passadas sejam utilizadas para construir carteiras
    \item As carteiras sejam aplicadas exclusivamente em períodos futuros
    \item Não haja contaminação temporal entre estimação e teste
    \item Os resultados reflitam performance realizável na prática
\end{enumerate}

O conceito de out-of-sample é fundamental para validação acadêmica, pois simula condições reais de investimento onde o futuro é desconhecido no momento da decisão.

\section{REBALANCEAMENTO DAS CARTEIRAS}

O rebalanceamento semestral utiliza janela móvel de estimação, recalculando parâmetros (médias, volatilidades, correlações) a cada período com base apenas em dados históricos disponíveis até aquele momento.

Os resultados principais são apresentados sem custos de transação para permitir comparação direta entre estratégias. Adicionalmente, custos de transação (15 basis points) e análise de sensibilidade são analisados no Apêndice A, apresentando também resultados líquidos para verificação de robustez dos achados.

\section{AVALIAÇÃO DE DESEMPENHO}

O desempenho das carteiras será avaliado por:

\begin{itemize}
    \item \textbf{Índice de Sharpe:} Avaliação do retorno excedente ajustado pela volatilidade total.
    \item \textbf{Sortino Ratio:} Avaliação do retorno excedente ajustado apenas pelo risco de perdas (volatilidade negativa).
\end{itemize}

As métricas serão calculadas:
\begin{itemize}
    \item Para cada semestre individualmente;
    \item E para o período consolidado 2018--2019.
\end{itemize}

\section{ANÁLISE DOS RESULTADOS}

Os resultados obtidos serão analisados de forma comparativa, considerando o desempenho de cada estratégia em diferentes métricas de risco-retorno. A análise será estruturada em três dimensões principais:

\begin{itemize}
    \item \textbf{Análise de desempenho:} comparação dos Índices de Sharpe, Sortino e Information Ratio entre as três estratégias;
    \item \textbf{Análise de risco:} avaliação da volatilidade e drawdown máximo de cada carteira;
    \item \textbf{Análise temporal:} verificação da consistência dos resultados ao longo dos períodos semestrais.
\end{itemize}

A interpretação dos resultados levará em consideração as características específicas do mercado brasileiro no período estudado, bem como as limitações metodológicas previamente identificadas.


\begin{table}[h]
\centering
\caption{Etapas da Pesquisa e Ferramentas Utilizadas}
\begin{tabular}{|p{4cm}|p{4cm}|p{4cm}|}
\hline
\textbf{Etapa} & \textbf{Descrição} & \textbf{Ferramentas} \\
\hline
Coleta de Dados & Extração de preços históricos da B3 & Python, Economática \\
\hline
Tratamento de Dados & Cálculo de retornos e estatísticas & pandas, NumPy \\
\hline
Construção de Carteiras & Implementação das três estratégias & scipy.optimize, pandas \\
\hline
Análise de Desempenho & Cálculo de métricas e comparação & NumPy, matplotlib \\
\hline
Visualização & Gráficos e tabelas comparativas & matplotlib, seaborn \\
\hline
\end{tabular}
\label{tab:etapas_pesquisa}
\end{table}

\section{FLUXOGRAMA METODOLÓGICO}

\begin{figure}[H]
\centering
\includegraphics[width=0.8\textwidth]{image.png}
\caption{Fluxograma da Metodologia}
\textit{Fonte: Elaborado pelo autor.}
\label{fig:fluxograma_metodologia}
\end{figure}

\section{CRONOGRAMA DE ATIVIDADES}

A execução deste trabalho seguirá um cronograma estruturado em 6 meses, conforme apresentado na Tabela \ref{tab:cronograma}.

\begin{table}[h]
\centering
\caption{Cronograma de Atividades do TCC I (6 meses)}
\begin{tabular}{|p{5cm}|c|c|c|c|c|c|}
\hline
\textbf{Atividades} & \textbf{Mês 1} & \textbf{Mês 2} & \textbf{Mês 3} & \textbf{Mês 4} & \textbf{Mês 5} & \textbf{Mês 6} \\
\hline
1. Revisão Bibliográfica e Referencial Teórico & X & X & & & & \\
\hline
2. Definição da Amostra de Ativos & X & & & & & \\
\hline
3. Coleta e Tratamento de Dados & & X & & & & \\
\hline
\quad • Extração de dados da base Economática & & X & & & & \\
\hline
\quad • Cálculo de retornos, volatilidades e covariâncias & & X & & & & \\
\hline
4. Implementação das Carteiras e Rebalanceamento & & & & X & & \\
\hline
\quad • Código Python (pandas, NumPy, scipy) & & & & X & & \\
\hline
\quad • Rebalanceamento semestral (jan e jul) & & & & X & & \\
\hline
5. Cálculo de Métricas e Análise Comparativa & & & & & X & \\
\hline
\quad • Sharpe e Sortino Ratio & & & & & X & \\
\hline
\quad • Gráficos e tabelas comparativas & & & & & X & \\
\hline
6. Redação de Resultados e Discussão & & & & & & X \\
\hline
7. Conclusão, Revisão Final e Entrega & & & & & & X \\
\hline
\end{tabular}
\label{tab:cronograma}
\end{table}

\section{LIMITAÇÕES DO ESTUDO}

Apesar do rigor metodológico adotado, este estudo apresenta algumas limitações que devem ser consideradas na análise dos resultados. Em primeiro lugar, os resultados principais não incorporam custos de transação, taxas, slippage e tributação nas operações, embora uma análise robusta desses fatores seja apresentada no Apêndice A para verificação da consistência dos achados. Além disso, a utilização de séries históricas de retornos pressupõe que padrões passados se mantenham representativos para o futuro, o que pode não se confirmar em mercados sujeitos a choques exógenos e mudanças estruturais. Outra limitação refere-se à escolha de apenas três estratégias de alocação, desconsiderando alternativas mais recentes como o modelo de Hierarchical Risk Parity ou abordagens baseadas em Machine Learning, que poderiam trazer novas perspectivas. Por fim, a definição da taxa livre de risco como o CDI médio anualizado simplifica a realidade de investimentos no Brasil, que apresenta múltiplos instrumentos de renda fixa com diferentes graus de risco e liquidez. Tais limitações, embora não invalidem os resultados, indicam caminhos para aprofundamentos em pesquisas futuras.