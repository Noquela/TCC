% ==================================================================
% 4 METODOLOGIA
% ==================================================================

\chapter{METODOLOGIA}

\section{CARACTERIZAÇÃO DA PESQUISA}

Este estudo apresenta características de pesquisa quantitativa, descritiva e comparativa, com abordagem empírica baseada em dados históricos do mercado financeiro brasileiro. O trabalho avalia o desempenho de três estratégias de alocação de ativos -- Equal Weight, Mean-Variance Optimization e Risk Parity -- em um ambiente controlado e metodologicamente rigoroso.

A estratégia metodológica adotada busca eliminar vieses comuns na literatura acadêmica, especialmente o look-ahead bias e o survivorship bias, através da implementação de critérios objetivos e transparentes para seleção de ativos e construção de carteiras.

\section{BASE DE DADOS E FONTE}

\subsection{Fonte dos Dados}

Os dados utilizados neste estudo foram obtidos exclusivamente da base da Economática, uma das principais provedoras de dados financeiros para o mercado brasileiro, amplamente utilizada em pesquisas acadêmicas e pela indústria financeira.

A base de dados compreende duas fontes principais:
\begin{itemize}
    \item \textbf{Arquivo de dados históricos:} Economatica-8900701390-20250812230945.xlsx -- contendo séries temporais de preços, volumes e indicadores financeiros;
    \item \textbf{Arquivo de classificação setorial:} economatica.xlsx -- contendo o mapeamento detalhado de ativos por setor econômico.
\end{itemize}

\subsection{Período de Análise}

O estudo utiliza um desenho metodológico rigoroso que separa claramente os períodos de seleção e teste para evitar look-ahead bias:

\begin{itemize}
    \item \textbf{Período de seleção (in-sample):} Janeiro de 2014 a dezembro de 2017 -- utilizado exclusivamente para aplicação dos critérios de seleção de ativos e cálculo de métricas de qualidade;
    \item \textbf{Período de teste (out-of-sample):} Janeiro de 2018 a dezembro de 2019 -- utilizado para avaliação da performance das estratégias, totalizando 24 observações mensais.
\end{itemize}

Esta separação temporal é fundamental para garantir que nenhuma informação do período de teste seja utilizada na seleção dos ativos, eliminando o look-ahead bias e tornando os resultados academicamente defensáveis.

\section{PROCESSO DE SELEÇÃO CIENTÍFICA DE ATIVOS}

\subsection{Universo Inicial}

A partir da base da Economática, foram identificados 507 ativos (sheets) disponíveis para análise. Deste universo, foram selecionados os primeiros 50 ativos para análise detalhada, resultando em 29 ativos com dados históricos válidos e completos para o período de estudo.

\subsection{Critérios Objetivos de Seleção}

O processo de seleção segue critérios quantitativos rigorosos, aplicados exclusivamente com base em dados do período 2014-2017:

\subsubsection{Critério 1: Completude de Dados}
Ativos devem apresentar pelo menos 70\% de dados válidos no período de seleção, garantindo robustez estatística das análises. Este critério eliminou 3 ativos da amostra inicial.

\subsubsection{Critério 2: Volatilidade Adequada}
Aplicação de filtro estatístico baseado no intervalo interquartil (IQR) para eliminar outliers de volatilidade:
\begin{itemize}
    \item Limite inferior: $Q_1 - 1,5 \times IQR$ ou 10\% (o que for maior)
    \item Limite superior: $Q_3 + 1,5 \times IQR$ ou 80\% (o que for menor)
\end{itemize}

Este critério eliminou 2 ativos com volatilidade extrema.

\subsubsection{Critério 3: Proxy de Liquidez}
Desenvolvimento de um indicador de liquidez baseado na variação média dos preços, onde maior estabilidade de preços indica menor liquidez:
$$\text{Proxy de Liquidez} = \frac{1}{\bar{\sigma}_{\text{preços}} + \epsilon}$$

Foram selecionados apenas ativos acima do percentil 30 desta métrica, eliminando 7 ativos com baixa liquidez.

\subsubsection{Critério 4: Dados Suficientes para Teste}
Garantia de pelo menos 20 meses de dados válidos no período out-of-sample (2018-2019) para robustez dos testes estatísticos.

\subsection{Diversificação Setorial}

Para evitar concentração setorial excessiva, foi aplicada restrição de máximo 2 ativos por setor econômico, baseada na classificação setorial da Economática. Este critério visa reduzir riscos sistemáticos e aumentar a representatividade da carteira.

A classificação setorial identificou 8 setores distintos na amostra final:
\begin{itemize}
    \item Bancos (2 ativos)
    \item Calçados (2 ativos) 
    \item Produtos diversos (1 ativo)
    \item Cervejas e refrigerantes (1 ativo)
    \item Exploração de imóveis (1 ativo)
    \item Tecidos vestuário e calçados (1 ativo)
    \item Serviços educacionais (1 ativo)
    \item Máquinas e equipamentos industriais (1 ativo)
\end{itemize}

\subsection{Score de Seleção Final}

Os ativos que passaram por todos os filtros anteriores foram ranqueados através de um score composto, calculado como:

$$\text{Score} = 0,4 \times \text{Liquidez Normalizada} + 0,3 \times \text{Cap. Mercado Normalizada} + 0,3 \times \text{Completude Normalizada}$$

onde cada componente foi normalizado pelo seu valor máximo na amostra.

\subsection{Ativos Selecionados}

O processo científico resultou na seleção de 10 ativos, apresentados na Tabela \ref{tab:ativos_selecionados} com seus respectivos scores de seleção e setores econômicos.

\begin{table}[htbp]
\centering
\caption{Ativos Selecionados por Critérios Científicos}
\label{tab:ativos_selecionados}
\begin{tabular}{|l|c|c|l|}
\hline
\textbf{Ativo} & \textbf{Score} & \textbf{Vol. 2014-2017} & \textbf{Setor} \\
\hline
ABEV3 & 0,802 & 12,4\% & Cervejas e refrigerantes \\
ALUP11 & 0,764 & 27,1\% & Energia elétrica \\
AMER3 & 0,762 & 57,0\% & Produtos diversos \\
ALOS3 & 0,743 & 31,3\% & Exploração de imóveis \\
ABCB4 & 0,682 & 34,2\% & Bancos \\
AZZA3 & 0,681 & 32,8\% & Tecidos vestuário e calçados \\
ALPA4 & 0,660 & 34,7\% & Calçados \\
BEES3 & 0,627 & 24,7\% & Serviços educacionais \\
B3SA3 & 0,620 & 29,3\% & Mercados de capitais \\
ALPA3 & 0,604 & 42,3\% & Calçados \\
\hline
\end{tabular}
\footnotesize
Fonte: Elaboração própria com dados da Economática.\\
Nota: Score baseado em liquidez (40\%), capitalização (30\%) e completude (30\%).
\end{table}

\section{TRATAMENTO E PREPARAÇÃO DOS DADOS}

\subsection{Frequência e Agregação}

Os dados diários de preços foram agregados em frequência mensal, utilizando o preço de fechamento do último dia útil de cada mês. Esta escolha metodológica visa:
\begin{itemize}
    \item Reduzir ruídos de alta frequência comuns em dados diários;
    \item Permitir análise de médio prazo mais adequada para estratégias de alocação;
    \item Facilitar comparações com a literatura acadêmica, que frequentemente utiliza dados mensais.
\end{itemize}

\subsection{Cálculo de Retornos}

Os retornos mensais foram calculados como:
$$r_{i,t} = \frac{P_{i,t} - P_{i,t-1}}{P_{i,t-1}}$$

onde $P_{i,t}$ representa o preço ajustado por proventos do ativo $i$ no final do mês $t$.

\subsection{Tratamento Rigoroso de Dados Faltantes}

Foi implementada uma metodologia em quatro etapas para tratamento de dados faltantes:

\subsubsection{Etapa 1: Critério de Exclusão}
Ativos com mais de 15\% de dados faltantes no período out-of-sample foram removidos da análise, garantindo robustez estatística.

\subsubsection{Etapa 2: Interpolação Linear}
Para gaps de até 2 meses consecutivos, aplicou-se interpolação linear entre os valores válidos adjacentes.

\subsubsection{Etapa 3: Forward e Backward Fill}
Para dados faltantes no início ou final das séries, aplicou-se propagação do último/primeiro valor válido (limitada a 1 mês).

\subsubsection{Etapa 4: Média Histórica}
Como último recurso, valores ainda faltantes foram substituídos pela média histórica do ativo no período válido.

Este processo garantiu que o dataset final não apresentasse valores faltantes, mantendo a integridade estatística das análises.

\section{ESTRATÉGIAS DE ALOCAÇÃO}

\subsection{Equal Weight (EW)}

A estratégia Equal Weight aloca o capital de forma igualitária entre todos os ativos da carteira:
$$w_i = \frac{1}{N}$$

onde $N$ é o número de ativos (10 neste estudo). Esta estratégia serve como benchmark por sua simplicidade e é amplamente utilizada na literatura acadêmica.

\subsection{Mean-Variance Optimization (MVO)}

A estratégia de Markowitz busca maximizar o Índice de Sharpe através de otimização matemática:

$$\max_{w} \frac{\boldsymbol{w}^T \boldsymbol{\mu} - r_f}{\sqrt{\boldsymbol{w}^T \boldsymbol{\Sigma} \boldsymbol{w}}}$$

sujeito às restrições:
\begin{align}
\sum_{i=1}^{N} w_i &= 1 \\
0 \leq w_i &\leq 0,4 \quad \forall i
\end{align}

onde $\boldsymbol{\mu}$ é o vetor de retornos esperados, $\boldsymbol{\Sigma}$ é a matriz de covariância, $r_f$ é a taxa livre de risco e $w_i$ são os pesos dos ativos.

A restrição de peso máximo de 40\% por ativo visa evitar concentração excessiva e tornar a estratégia implementável na prática. A otimização é realizada utilizando o método SLSQP da biblioteca scipy.optimize do Python.

\subsection{Risk Parity (ERC)}

A estratégia Equal Risk Contribution busca equalizar a contribuição marginal de risco de cada ativo ao risco total da carteira. Matematicamente, busca-se:

$$\frac{\partial \sigma_p}{\partial w_i} w_i = \frac{\sigma_p}{N} \quad \forall i$$

onde $\sigma_p$ é a volatilidade da carteira.

A implementação utiliza algoritmo iterativo que converge para a solução onde:
$$w_i \times (\boldsymbol{\Sigma} \boldsymbol{w})_i = \frac{\sigma_p}{N} \quad \forall i$$

O processo iterativo continua até que a diferença entre iterações seja inferior a $10^{-6}$ ou seja atingido o limite de 1.000 iterações.

\section{MÉTRICAS DE AVALIAÇÃO}

\subsection{Métricas de Performance}

Para cada estratégia, são calculadas as seguintes métricas:

\subsubsection{Retorno Anualizado}
$$\bar{r}_{\text{anual}} = \bar{r}_{\text{mensal}} \times 12$$

\subsubsection{Volatilidade Anualizada}
$$\sigma_{\text{anual}} = \sigma_{\text{mensal}} \times \sqrt{12}$$

\subsubsection{Índice de Sharpe}
$$\text{Sharpe} = \frac{\bar{r}_{\text{anual}} - r_f}{\sigma_{\text{anual}}}$$

\subsubsection{Índice de Sortino}
$$\text{Sortino} = \frac{\bar{r}_{\text{anual}} - r_f}{\sigma_{\text{downside}}}$$

onde $\sigma_{\text{downside}}$ é a volatilidade calculada apenas sobre retornos negativos.

\subsubsection{Maximum Drawdown}
$$\text{MDD} = \min_{t} \left( \frac{\text{Valor}_t - \text{Pico Anterior}}{\text{Pico Anterior}} \right)$$

\subsection{Taxa Livre de Risco}

Foi utilizada como taxa livre de risco o CDI (Certificado de Depósito Interbancário) médio do período 2018-2019, equivalente a 6,24\% ao ano, refletindo a realidade do mercado brasileiro no período de análise.

\section{TESTES DE SIGNIFICÂNCIA ESTATÍSTICA}

Para validar estatisticamente as diferenças de performance entre as estratégias, foi implementado o teste de Jobson-Korkie (1981), específico para comparação de Índices de Sharpe.

O teste verifica as hipóteses:
\begin{align}
H_0&: \text{Sharpe}_1 = \text{Sharpe}_2 \\
H_1&: \text{Sharpe}_1 \neq \text{Sharpe}_2
\end{align}

A estatística do teste é:
$$t = \frac{\hat{S}_1 - \hat{S}_2}{\sqrt{\text{Var}(\hat{S}_1 - \hat{S}_2)}}$$

onde a variância da diferença considera a correlação entre as séries de retornos das estratégias comparadas.

\section{FERRAMENTAS E IMPLEMENTAÇÃO}

O estudo foi implementado integralmente na linguagem Python, utilizando as seguintes bibliotecas principais:

\begin{itemize}
    \item \textbf{pandas:} manipulação e análise de dados estruturados;
    \item \textbf{numpy:} cálculos matriciais e operações matemáticas;
    \item \textbf{scipy:} otimização matemática e testes estatísticos;
    \item \textbf{openpyxl:} leitura dos arquivos Excel da Economática.
\end{itemize}

Todo o código desenvolvido segue princípios de reprodutibilidade científica, com documentação detalhada e modularização que permite replicação dos resultados.

\section{LIMITAÇÕES METODOLÓGICAS}

É importante reconhecer algumas limitações inerentes ao desenho metodológico adotado:

\begin{itemize}
    \item \textbf{Período de análise:} O período out-of-sample de 24 meses, embora adequado estatisticamente, é relativamente curto para conclusões de longo prazo;
    \item \textbf{Tamanho da amostra:} A análise de 10 ativos, embora cientificamente selecionados, representa uma amostra moderada do universo de investimentos brasileiro;
    \item \textbf{Custos de transação:} Não foram incorporados custos operacionais como corretagem, custódia e impacto de mercado;
    \item \textbf{Liquidez:} A proxy de liquidez desenvolvida, embora objetiva, não captura perfeitamente a facilidade real de negociação dos ativos.
\end{itemize}

Essas limitações são comuns na literatura acadêmica de finanças e não comprometem a validade científica dos resultados obtidos, servindo como direcionadores para estudos futuros.