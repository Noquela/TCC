% ==================================================================
% 5 DISCUSSÃO
% ==================================================================

\chapter{DISCUSSÃO}

Este capítulo interpreta os resultados empíricos apresentados no capítulo anterior, contextualizando-os com a literatura acadêmica e discutindo suas implicações teóricas e práticas. A análise concentra-se na compreensão dos mecanismos que explicam a performance observada das três estratégias durante o período 2018-2019.

\section{INTERPRETAÇÃO DOS RESULTADOS PRINCIPAIS}

\subsection{Performance da Estratégia de Markowitz}

Os resultados demonstram que a otimização de Markowitz apresentou performance superior durante o período analisado, com Sharpe Ratio de 1,858, retorno anualizado de 42,45\% e maximum drawdown de apenas -14,61\%. Esta performance contradiz parcialmente a literatura que documenta dificuldades práticas da otimização clássica (MICHAUD, 1989; DEMIGUEL; GARLAPPI; UPPAL, 2009).

Três fatores podem explicar esta performance superior. Primeiro, a seleção científica de ativos com base em critérios objetivos (momentum, volatilidade, drawdown e downside deviation) pode ter reduzido significativamente os problemas tradicionais de "error maximization" identificados por Michaud (1989). Quando aplicada a ativos pré-filtrados por qualidade, a sensibilidade da otimização a erros de estimação é mitigada.

Segundo, o período 2018-2019 caracterizou-se por alta volatilidade e dispersão significativa entre performances individuais dos ativos no mercado brasileiro. Durante regimes de mercado com maior dispersão, estratégias que conseguem identificar e concentrar capital nos ativos mais promissores tendem a superar estratégias de diversificação mecânica (CHOPRA; ZIEMBA, 1993).

Terceiro, o universo reduzido de 10 ativos selecionados pode favorecer estratégias de concentração seletiva sobre estratégias de diversificação ampla. As estratégias Risk Parity e Equal Weight foram originalmente desenvolvidas para universos maiores, onde a diversificação oferece benefícios mais claros (MAILLARD; RONCALLI; TEILETCHE, 2010).

\subsection{Performance da Estratégia Equal Weight}

Equal Weight apresentou performance intermediária, com Sharpe Ratio de 1,197, retorno anualizado de 29,84\% e maximum drawdown de -18,88\%. Estes resultados são consistentes com a literatura que documenta a eficácia de estratégias de peso igual como benchmark robusto (DEMIGUEL; GARLAPPI; UPPAL, 2009).

A simplicidade operacional e a ausência de dependência de estimações de parâmetros conferem à estratégia Equal Weight resistência a erros de modelo e instabilidade de inputs. Esta característica explica sua performance consistente, embora não excepcional, durante o período analisado.

A estratégia demonstrou resiliência particular durante eventos de estresse, como a greve dos caminhoneiros em maio de 2018 e o período eleitoral de setembro-outubro de 2018, mantendo trajetória de crescimento relativamente estável comparada às demais estratégias.

\subsection{Performance da Estratégia Risk Parity}

Risk Parity apresentou performance inferior às expectativas, com Sharpe Ratio de 1,209, retorno anualizado de 28,75\% e maximum drawdown de -18,19\%. Este resultado contrasta com estudos que demonstram superioridade de estratégias de paridade de risco em diversos contextos (MAILLARD; RONCALLI; TEILETCHE, 2010; QIAN, 2005).

A performance inferior pode ser explicada por limitações específicas em universos de alta qualidade. Risk Parity foi desenvolvida para funcionar eficazmente em universos diversos com ampla dispersão de características de risco. Em universos de ativos de alta qualidade, onde a dispersão de volatilidades é menor e todos os ativos apresentam características fundamentalmente sólidas, a equalização de contribuições de risco pode levar à sub-otimização.

Adicionalmente, a filosofia de Risk Parity de evitar concentração pode ser contraproducente quando aplicada a ativos verdadeiramente superiores. O algoritmo, ao forçar contribuições de risco iguais, pode reduzir exposição a ativos excepcionais em favor de diversificação mecânica.

\section{VALIDAÇÃO ESTATÍSTICA DOS RESULTADOS}

Os testes de significância estatística (Jobson-Korkie) confirmam que as diferenças observadas entre as estratégias são estatisticamente significativas. A comparação entre Markowitz e Risk Parity apresentou significância ao nível de 1\% (p-valor < 0,001), assim como a comparação entre Markowitz e Equal Weight (p-valor < 0,001). Ambas as diferenças são estatisticamente robustas.

Estas evidências estatísticas fortalecem a confiança nos resultados observados, indicando que as diferenças de performance não são devidas ao acaso amostral. Contudo, é importante reconhecer que o período de análise de 24 meses oferece base empírica sólida mas não definitiva para conclusões generalizáveis.

\section{CONTEXTUALIZAÇÃO COM A LITERATURA}

\subsection{Convergência e Divergência com Estudos Anteriores}

Os resultados divergem parcialmente de estudos internacionais que documentam dificuldades sistemáticas da otimização de Markowitz (DEMIGUEL; GARLAPPI; UPPAL, 2009) e superioridade frequente de estratégias alternativas como Risk Parity (MAILLARD; RONCALLI; TEILETCHE, 2010).

Esta divergência pode ser explicada por diferenças metodológicas fundamentais. Estudos internacionais típicos utilizam universos amplos (50-500 ativos), seleção baseada em capitalização de mercado, períodos longos (10-30 anos) e mercados desenvolvidos com maior eficiência informacional.

Em contraste, este estudo emprega universo concentrado (10 ativos), seleção baseada em critérios científicos de qualidade, período específico (2 anos) e mercado emergente com características peculiares. Estas diferenças metodológicas podem explicar a inversão de resultados, sugerindo que contexto é tão importante quanto metodologia.

\subsection{Contribuição à Literatura Nacional}

Este estudo contribui para a escassa literatura brasileira sobre estratégias de alocação de ativos, oferecendo evidências empíricas baseadas em dados reais do mercado nacional. A aplicação de metodologias científicas de seleção de ativos ao contexto brasileiro preenche lacuna importante na literatura acadêmica nacional.

Os resultados sugerem que conclusões baseadas em mercados desenvolvidos podem não se aplicar diretamente ao contexto brasileiro, enfatizando a necessidade de pesquisa localizada e consideração de especificidades do mercado doméstico.

\section{LIMITAÇÕES DO ESTUDO}

\subsection{Limitações Temporais}

O período de análise de 24 meses (janeiro 2018 - dezembro 2019) oferece evidência empírica inicial mas relativamente limitada para conclusões definitivas sobre eficácia das estratégias. Períodos mais longos seriam necessários para maior robustez estatística e confiança nas conclusões.

O período específico analisado foi caracterizado por eventos extraordinários no mercado brasileiro, incluindo eleições presidenciais, reformas estruturais e alta volatilidade política. Embora estes eventos ofereçam teste rigoroso para as estratégias, podem limitar a generalização dos resultados para períodos mais estáveis.

\subsection{Limitações de Universo}

O universo de 10 ativos, embora cientificamente selecionado, representa amostra pequena comparada aos típicos 50-500 ativos utilizados na prática por gestores institucionais. A generalização dos resultados para universos maiores requer validação adicional.

A concentração em ativos de alta qualidade, embora metodologicamente rigorosa, pode não refletir a realidade prática onde gestores frequentemente trabalham com universos mais amplos e diversos em termos de qualidade.

\subsection{Limitações Geográficas}

Os resultados são específicos ao mercado brasileiro durante 2018-2019. A aplicação das conclusões a outros mercados emergentes ou desenvolvidos requer investigação adicional, considerando diferenças em estrutura de mercado, regulação e comportamento dos investidores.

\section{IMPLICAÇÕES PRÁTICAS}

\subsection{Para Gestores de Recursos}

Os resultados sugerem que investimento significativo em metodologias rigorosas de seleção de ativos pode gerar mais valor que sofisticação excessiva em técnicas de alocação. Gestores podem considerar rebalanceamento de recursos entre processos de seleção e alocação.

A eficácia demonstrada da otimização de Markowitz quando aplicada a ativos cuidadosamente selecionados sugere reconsideração de seu uso, especialmente em contextos onde qualidade dos ativos é controlável através de processos rigorosos de due diligence.

\subsection{Para Investidores Institucionais}

A importância crítica da qualidade na seleção de ativos implica necessidade de due diligence rigoroso na avaliação de gestores, focando não apenas em metodologias de alocação mas também na qualidade dos processos de seleção de ativos.

Investidores podem considerar diversificação não apenas entre classes de ativos, mas entre diferentes filosofias de seleção e alocação, reconhecendo que eficácia é contextual e dependente das características do universo de ativos.

\section{DIREÇÕES PARA PESQUISA FUTURA}

\subsection{Extensões Temporais e Geográficas}

Pesquisas futuras devem validar os resultados em diferentes períodos históricos e mercados geográficos para verificar a robustez e generalização das conclusões. Análises de períodos mais longos (5-10 anos) ofereceriam maior confiança estatística.

\subsection{Universos e Metodologias Alternativas}

Investigações com diferentes tamanhos de universo (5, 15, 20 ativos) ajudariam a compreender como escala afeta a eficácia relativa das estratégias. Adicionalmente, exploração de critérios alternativos de seleção de ativos poderia oferecer insights sobre a robustez da abordagem.

\subsection{Integração de Novas Tecnologias}

A aplicação de técnicas de machine learning tanto para seleção quanto para alocação representa fronteira promissora para pesquisa, potencialmente oferecendo melhorias sobre as metodologias tradicionais analisadas neste estudo.

\section{SÍNTESE DA DISCUSSÃO}

Esta discussão demonstra que os resultados empíricos, embora surpreendentes em alguns aspectos, são explicáveis através de mecanismos teóricos sólidos e contextualização adequada. A performance superior de Markowitz, intermediária de Equal Weight e inferior de Risk Parity reflete interação complexa entre qualidade dos ativos, características do período e especificidades metodológicas.

As implicações práticas sugerem rebalanceamento na importância atribuída à seleção versus alocação de ativos, enquanto as limitações reconhecidas orientam direções produtivas para pesquisa futura. Este estudo contribui para a literatura nacional oferecendo evidências empíricas baseadas no mercado brasileiro e metodologias científicas rigorosas.