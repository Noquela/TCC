% ==================================================================
% 6 CONCLUSÃO
% ==================================================================

\chapter{CONCLUSÃO}

Este trabalho analisou a eficácia de três estratégias de alocação de ativos aplicadas ao mercado acionário brasileiro durante o período 2018-2019. A pesquisa comparou a performance da otimização de Markowitz, estratégia Equal Weight e estratégia Risk Parity em uma carteira de 10 ações selecionadas através de critérios científicos objetivos.

\section{SÍNTESE DOS RESULTADOS}

\subsection{Achados Principais}

Os resultados empíricos demonstram performance diferenciada entre as três estratégias analisadas. A otimização de Markowitz apresentou performance superior, com Sharpe Ratio de 1,858, retorno anualizado de 42,45\% e maximum drawdown de -14,61\%. Equal Weight apresentou performance intermediária (Sharpe Ratio: 1,197; retorno: 29,84\%; drawdown: -18,88\%), enquanto Risk Parity apresentou performance mais modesta (Sharpe Ratio: 1,209; retorno: 28,75\%; drawdown: -18,19\%).

A validação estatística através do teste de Jobson-Korkie confirma significância das diferenças observadas. A superioridade de Markowitz sobre Risk Parity é estatisticamente significativa ao nível de 1\% (p-valor < 0,001), assim como a diferença entre Markowitz e Equal Weight (p-valor < 0,001). Ambas as diferenças são estatisticamente robustas.

\subsection{Resposta à Questão de Pesquisa}

A questão central desta pesquisa investigava qual estratégia de alocação de ativos proporcionaria melhor relação risco-retorno no mercado brasileiro durante período de alta volatilidade. Os resultados indicam que a otimização de Markowitz, quando aplicada a ativos cientificamente selecionados, apresentou a melhor relação risco-retorno durante o período analisado.

Este achado contrasta parcialmente com estudos internacionais que documentam dificuldades sistemáticas da otimização clássica. A divergência pode ser explicada pela qualidade superior dos ativos selecionados através de critérios científicos, que reduz os problemas tradicionais de "error maximization" que afetam a otimização de Markowitz.

\subsection{Cumprimento dos Objetivos Específicos}

\textbf{Objetivo 1 - Implementar metodologia científica de seleção:} Cumprido através do desenvolvimento de score composto baseado em momentum, volatilidade, maximum drawdown e downside deviation, aplicado ao período 2014-2017 com critérios rigorosos de liquidez.

\textbf{Objetivo 2 - Comparar estratégias empiricamente:} Cumprido através da implementação das três estratégias durante 2018-2019, com análise abrangente de métricas de performance, risco e características de implementação.

\textbf{Objetivo 3 - Analisar performance durante volatilidade:} Cumprido através da análise detalhada do comportamento das estratégias durante eventos específicos (greve dos caminhoneiros, eleições presidenciais) e períodos de estresse.

\textbf{Objetivo 4 - Fornecer recomendações práticas:} Cumprido através da discussão de implicações para gestores, investidores institucionais e desenvolvimento de produtos.

\section{CONTRIBUIÇÕES DO ESTUDO}

\subsection{Contribuição Acadêmica}

Este estudo contribui para a literatura brasileira de finanças oferecendo evidência empírica sistemática sobre estratégias de alocação aplicadas especificamente ao mercado nacional. A implementação rigorosa de metodologia out-of-sample elimina look-ahead bias comum em estudos da área.

A pesquisa demonstra que contexto importa significativamente para eficácia de estratégias de alocação. Os resultados sugerem que conclusões baseadas em mercados desenvolvidos podem não se aplicar diretamente ao contexto brasileiro, enfatizando a necessidade de pesquisa localizada.

\subsection{Contribuição Metodológica}

O framework de seleção científica de ativos desenvolvido oferece abordagem objetiva e replicável para curadoria de universos de investimento. A combinação de critérios quantitativos (momentum, volatilidade, drawdown, downside) com filtros rigorosos de liquidez representa melhoria sobre práticas tradicionais baseadas apenas em capitalização de mercado.

A metodologia integrada que combina seleção científica com comparação rigorosa de estratégias de alocação oferece modelo para pesquisas futuras em mercados emergentes.

\subsection{Contribuição Prática}

Os resultados sugerem que investimento em processos rigorosos de seleção de ativos pode ser tão importante quanto sofisticação em técnicas de alocação. Esta descoberta tem implicações práticas significativas para gestores de recursos e investidores institucionais.

A eficácia demonstrada da otimização de Markowitz quando aplicada a ativos de alta qualidade sugere reconsideração de seu uso prático, especialmente em contextos onde qualidade dos ativos é controlável.

\section{LIMITAÇÕES DO ESTUDO}

\subsection{Limitações Temporais}

O período de análise de 24 meses (janeiro 2018 - dezembro 2019) oferece evidência empírica sólida mas relativamente limitada para conclusões definitivas sobre eficácia das estratégias. Períodos mais longos seriam necessários para maior robustez estatística e confirmação dos resultados em diferentes regimes de mercado.

O período específico analisado foi caracterizado por eventos extraordinários no mercado brasileiro, incluindo eleições presidenciais e alta volatilidade política. Embora estes eventos ofereçam teste rigoroso para as estratégias, podem limitar a generalização dos resultados.

\subsection{Limitações de Escopo}

O universo de 10 ativos, embora cientificamente selecionado, representa amostra pequena comparada à prática institucional típica. A generalização dos resultados para universos maiores requer validação adicional.

A concentração em ações de alta liquidez do mercado brasileiro pode não refletir adequadamente a diversidade completa de oportunidades de investimento disponíveis no mercado nacional.

\subsection{Limitações Geográficas}

Os resultados são específicos ao mercado brasileiro durante 2018-2019. A aplicação das conclusões a outros mercados emergentes ou desenvolvidos requer investigação adicional, considerando diferenças estruturais, regulatórias e comportamentais.

\section{IMPLICAÇÕES E RECOMENDAÇÕES}

\subsection{Para Gestores de Recursos}

Os resultados sugerem rebalanceamento na alocação de recursos entre processos de seleção e alocação de ativos. Investimento significativo em metodologias rigorosas de seleção pode gerar mais valor que sofisticação excessiva em técnicas de alocação.

Gestores podem reconsiderar o uso da otimização de Markowitz quando aplicada a universos cuidadosamente curados, especialmente em contextos onde qualidade dos ativos é controlável através de due diligence rigoroso.

\subsection{Para Investidores Institucionais}

A importância da qualidade na seleção de ativos implica necessidade de avaliação rigorosa dos processos de seleção utilizados por gestores, complementando a análise tradicional de metodologias de alocação.

Investidores podem considerar diversificação entre diferentes filosofias de seleção e alocação, reconhecendo que eficácia é contextual e dependente das características do universo de ativos.

\subsection{Para Desenvolvimento de Produtos}

Os resultados sugerem oportunidade para desenvolvimento de produtos de investimento que integram seleção científica de ativos com otimização sofisticada, aproveitando os benefícios demonstrados desta abordagem combinada.

\section{DIREÇÕES PARA PESQUISA FUTURA}

\subsection{Extensões Temporais}

Pesquisas futuras devem validar os resultados em períodos mais extensos (5-10 anos) e diferentes ciclos econômicos para verificar robustez e generalização das conclusões. Análises de diferentes regimes de mercado ofereceriam insights valiosos sobre condições ótimas para cada estratégia.

\subsection{Extensões Metodológicas}

Investigações com diferentes tamanhos de universo (5, 15, 20 ativos) ajudariam a compreender como escala afeta eficácia relativa das estratégias. Exploração de critérios alternativos de seleção (fatores fundamentalistas, técnicos, ESG) poderia oferecer insights sobre robustez da abordagem.

\subsection{Extensões Geográficas}

Aplicação da metodologia a outros mercados emergentes (México, Colômbia, Chile) e mercados desenvolvidos permitiria avaliar generalização dos achados e identificar características específicas que influenciam eficácia das estratégias.

\subsection{Integração Tecnológica}

A aplicação de técnicas de machine learning tanto para seleção quanto para alocação representa fronteira promissora, potencialmente oferecendo melhorias sobre as metodologias tradicionais analisadas.

\section{CONSIDERAÇÕES FINAIS}

Este estudo demonstra que estratégias de alocação de ativos apresentam eficácia condicional, dependente significativamente da qualidade dos ativos subjacentes e características do período analisado. A performance superior da otimização de Markowitz, quando aplicada a ativos cientificamente selecionados, sugere que a integração de processos rigorosos de seleção com técnicas sofisticadas de otimização pode oferecer valor superior a abordagens que focam exclusivamente em uma ou outra dimensão.

Os resultados contribuem para a literatura brasileira de finanças oferecendo evidência empírica sistemática e metodologia replicável para pesquisas futuras. Embora limitados ao contexto específico analisado, os achados sugerem direções produtivas para desenvolvimento de práticas mais eficazes na gestão de investimentos em mercados emergentes.

A pesquisa demonstra que questões aparentemente simples sobre eficácia de estratégias de investimento requerem análise cuidadosa e contextualizada. O sucesso relativo das estratégias depende não apenas de suas características intrínsecas, mas também da qualidade dos inputs, características do mercado e período de implementação.

Este trabalho oferece base sólida para pesquisas futuras e desenvolvimento de práticas mais rigorosas na indústria de gestão de recursos brasileira, contribuindo para o aprimoramento do mercado de capitais nacional através de evidência científica sistemática.