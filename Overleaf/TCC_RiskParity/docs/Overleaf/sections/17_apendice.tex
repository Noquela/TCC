% ==================================================================
% APÊNDICE A - PESOS DAS ESTRATÉGIAS E REPRODUTIBILIDADE
% ==================================================================

\chapter{APÊNDICE A - DETALHES TÉCNICOS DAS ESTRATÉGIAS}

\section{ALOCAÇÃO DE PESOS POR ESTRATÉGIA}

\subsection{Pesos Iniciais das Carteiras (Janeiro 2018)}

A Tabela \ref{tab:pesos_estrategias} apresenta os pesos iniciais de cada ativo nas três estratégias implementadas, calculados com base nos dados históricos de 2016-2017.

\begin{table}[H]
\centering
\caption{Alocação de Pesos por Estratégia - Janeiro 2018}
\begin{tabular}{|l|r|r|r|}
\hline
\textbf{Ativo} & \textbf{Equal Weight} & \textbf{Markowitz} & \textbf{Risk Parity} \\
& \textbf{(\%)} & \textbf{(\%)} & \textbf{(\%)} \\
\hline
PETR4 & 10,0 & 12,1 & 8,5 \\
\hline
VALE3 & 10,0 & 15,8 & 9,2 \\
\hline
ITUB4 & 10,0 & 18,3 & 14,7 \\
\hline
BBDC4 & 10,0 & 14,2 & 13,1 \\
\hline
ABEV3 & 10,0 & 11,6 & 15,8 \\
\hline
B3SA3 & 10,0 & 8,4 & 9,3 \\
\hline
WEGE3 & 10,0 & 7,9 & 12,4 \\
\hline
RENT3 & 10,0 & 5,2 & 8,7 \\
\hline
LREN3 & 10,0 & 3,8 & 4,9 \\
\hline
ELET3 & 10,0 & 2,7 & 3,4 \\
\hline
\textbf{Total} & \textbf{100,0} & \textbf{100,0} & \textbf{100,0} \\
\hline
\end{tabular}

\textit{Fonte: Elaborado pelo autor com base em dados da Economática (2016-2017).}
\label{tab:pesos_estrategias}
\end{table}

\subsection{Rebalanceamento e Turnover}

As carteiras foram rebalanceadas semestralmente, gerando os seguintes níveis de turnover:

\begin{table}[H]
\centering
\caption{Turnover por Rebalanceamento Semestral}
\begin{tabular}{|l|r|r|r|}
\hline
\textbf{Rebalanceamento} & \textbf{Equal Weight} & \textbf{Markowitz} & \textbf{Risk Parity} \\
& \textbf{(\%)} & \textbf{(\%)} & \textbf{(\%)} \\
\hline
Julho 2018 & 15,2 & 42,8 & 28,7 \\
\hline
Janeiro 2019 & 18,1 & 38,9 & 31,2 \\
\hline
Julho 2019 & 16,8 & 45,1 & 29,8 \\
\hline
\textbf{Médio} & \textbf{16,7} & \textbf{42,3} & \textbf{29,9} \\
\hline
\end{tabular}

\textit{Nota: Turnover calculado como soma dos valores absolutos das mudanças de peso, dividido por 2.}
\label{tab:turnover_rebalanceamento}
\end{table}

\section{LIMITES PRÁTICOS IMPLEMENTADOS}

As seguintes restrições foram aplicadas durante a otimização:

\begin{itemize}
    \item \textbf{Sem vendas a descoberto:} $w_i \geq 0$ para todos os ativos
    \item \textbf{Peso mínimo:} $w_i \geq 0,1\%$ para garantir diversificação mínima
    \item \textbf{Soma dos pesos:} $\sum_{i=1}^{n} w_i = 1$
    \item \textbf{Limite setorial:} Máximo 3 ativos por setor econômico (aplicado na seleção ex-ante)
    \item \textbf{Sem limite individual de teto:} Permitindo concentração natural conforme otimização
\end{itemize}

\section{CUSTOS DE TRANSAÇÃO E SLIPPAGE}

\subsection{Cenário Base: Custos de Transação}

Aplicação de custos de transação de 15 basis points (0,15\%) por operação:

\begin{table}[H]
\centering
\caption{Performance Líquida com Custos de Transação (15 bps)}
\begin{tabular}{|l|r|r|r|r|}
\hline
\textbf{Estratégia} & \textbf{Ret. Bruto} & \textbf{Custos} & \textbf{Ret. Líquido} & \textbf{Impacto} \\
& \textbf{(\%)} & \textbf{(\%)} & \textbf{(\%)} & \textbf{(bps)} \\
\hline
Equal Weight & 16,2 & 0,5 & 15,7 & 50 \\
\hline
Markowitz & 29,3 & 1,3 & 28,0 & 130 \\
\hline
Risk Parity & 18,7 & 0,9 & 17,8 & 90 \\
\hline
\end{tabular}

\textit{Nota: Custos calculados com base no turnover médio e frequência semestral.}
\label{tab:custos_transacao}
\end{table}

\subsection{Análise de Sensibilidade ao Slippage}

\begin{table}[H]
\centering
\caption{Sensibilidade a Diferentes Níveis de Custos}
\begin{tabular}{|l|r|r|r|}
\hline
\textbf{Cenário} & \textbf{Equal Weight} & \textbf{Markowitz} & \textbf{Risk Parity} \\
& \textbf{Ret. Líquido (\%)} & \textbf{Ret. Líquido (\%)} & \textbf{Ret. Líquido (\%)} \\
\hline
10 bps & 15,9 & 28,5 & 18,1 \\
\hline
15 bps (base) & 15,7 & 28,0 & 17,8 \\
\hline
30 bps & 15,2 & 26,7 & 17,1 \\
\hline
\end{tabular}

\textit{Conclusão: Markowitz mantém superioridade mesmo com custos elevados.}
\label{tab:sensibilidade_custos}
\end{table}

\section{REPRODUTIBILIDADE E VERSIONAMENTO}

\subsection{Ambiente Computacional}

\textbf{Versões das principais bibliotecas:}
\begin{itemize}
    \item Python 3.9.7
    \item pandas 1.3.4  
    \item numpy 1.21.2
    \item scipy 1.7.3
    \item cvxpy 1.2.0
    \item matplotlib 3.4.3
\end{itemize}

\subsection{Ordem de Execução}

Para reproduzir os resultados:

\begin{enumerate}
    \item \texttt{python src/economatica\_loader.py} - Seleção ex-ante de ativos
    \item \texttt{python src/ibovespa\_real\_loader.py} - Carregamento benchmark B3
    \item \texttt{python src/portfolio\_analysis\_real.py} - Análise das estratégias
    \item \texttt{python src/calculate\_real\_stats.py} - Estatísticas individuais
    \item \texttt{python src/generate\_real\_charts\_fixed.py} - Geração dos gráficos
\end{enumerate}

\subsection{Dados e Paths}

\textbf{Estrutura de diretórios requerida:}
\begin{verbatim}
TCC_RiskParity/
|-- data/DataBase/
|   |-- Evolucao_Diaria.csv     # Ibovespa 2019
|   |-- Evolucao_Diaria (1).csv # Ibovespa 2018  
|   +-- [arquivos Economática]
|-- src/                        # Scripts Python
|-- results/                    # Outputs
+-- docs/Overleaf/              # Documento LaTeX
\end{verbatim}

\textbf{Arquivos de output gerados:}
\begin{itemize}
    \item \texttt{results/selected\_assets\_2017.json} - Lista de ativos selecionados
    \item \texttt{results/real\_returns\_data.csv} - Retornos mensais
    \item \texttt{results/detailed\_portfolio\_results.json} - Métricas das estratégias
    \item \texttt{docs/Overleaf/images/} - Figuras geradas automaticamente
\end{itemize}