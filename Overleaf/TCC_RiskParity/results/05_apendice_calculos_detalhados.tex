% ==================================================================
% APÊNDICE - CÁLCULOS DETALHADOS PARA VALIDAÇÃO ACADÊMICA
% ==================================================================

\appendix
\chapter{CÁLCULOS DETALHADOS E VALIDAÇÃO CIENTÍFICA}

\section{INTRODUÇÃO AOS CÁLCULOS DETALHADOS}

Este apêndice apresenta todos os cálculos realizados neste estudo com detalhamento completo, permitindo validação independente e reprodução exata dos resultados por outros pesquisadores. Todos os valores, fórmulas e procedimentos são documentados com transparência científica total.

\textbf{Estrutura do Apêndice:}
\begin{enumerate}
    \item Processo detalhado de seleção científica dos ativos
    \item Cálculo das métricas de liquidez da Economática
    \item Algoritmo do score de seleção composto
    \item Matriz de retornos mensais e tratamento de dados
    \item Estimação da matriz de covariância
    \item Implementação da otimização de Markowitz
    \item Algoritmo de Equal Risk Contribution
    \item Cálculo das métricas de performance
    \item Testes de significância estatística
\end{enumerate}

\section{PROCESSO DE SELEÇÃO CIENTÍFICA DOS ATIVOS}

\subsection{Critérios de Liquidez Implementados}

A seleção dos ativos seguiu critérios rigorosos baseados em dados da Economática. Os filtros foram aplicados sequencialmente no período 2014-2017:

\subsection{Ativos Selecionados e Suas Características}

Os 10 ativos que passaram em todos os filtros científicos foram:

\begin{table}[H]
\centering
\caption{Ativos Selecionados - Características Detalhadas}
\begin{tabular}{|l|c|c|c|c|}
\hline
\textbf{Ativo} & \textbf{Volume R\$ (M)} & \textbf{Q Negs/dia} & \textbf{Presença \%} & \textbf{Score} \\
\hline
LREN3 & 71.4 & 7815 & 94.9\% & 0.727 \\
WEGE3 & 29.1 & 6070 & 94.9\% & 0.697 \\
ABEV3 & 224.8 & 23582 & 94.9\% & 0.690 \\
AZZA3 & 6.2 & 756 & 94.9\% & 0.680 \\
ALPA4 & 6.6 & 1496 & 94.9\% & 0.557 \\
RENT3 & 41.9 & 5434 & 94.9\% & 0.553 \\
ITUB4 & 436.7 & 27194 & 94.9\% & 0.537 \\
ALUP11 & 6.1 & 1367 & 94.9\% & 0.530 \\
B3SA3 & 169.1 & 21035 & 94.9\% & 0.527 \\
BBDC4 & 295.4 & 22858 & 94.9\% & 0.410 \\
\hline
\end{tabular}
\end{table}

\textbf{Validação dos Critérios:}
\begin{itemize}
    \item Todos os ativos atendem Volume ≥ R\$ 5M/dia
    \item Todos os ativos atendem Q Negs ≥ 500/dia
    \item Todos os ativos atendem Presença ≥ 90\%
    \item Diversificação: 7 setores econômicos distintos
\end{itemize}

\section{CÁLCULO DAS MÉTRICAS DE LIQUIDEZ}

\subsection{Volume Financeiro Médio Diário}

Para cada ativo $i$, o volume financeiro médio diário foi calculado como:

\begin{equation}
\text{Volume}\$_{i} = \frac{1}{T} \sum_{t=1}^{T} \text{Volume}\$_{i,t}
\end{equation}

onde $T$ é o número de dias no período 2014-2017 e Volume\$_{i,t} é o volume financeiro do ativo $i$ no dia $t$ conforme reportado pela Economática.

\subsection{Quantidade de Negócios por Dia}

A quantidade média de negócios por dia foi calculada como:

\begin{equation}
\text{Q Negs}_{i} = \frac{1}{T} \sum_{t=1}^{T} \text{Q Negs}_{i,t}
\end{equation}

onde Q Negs_{i,t} é a quantidade de negócios do ativo $i$ no dia $t$.

\subsection{Presença em Bolsa}

O percentual de dias com negociação foi calculado como:

\begin{equation}
\text{Presença}_{i} = \frac{\text{Dias com Volume}\$_{i,t} > 0}{\text{Total de dias úteis}} \times 100\%
\end{equation}

\section{ALGORITMO DO SCORE DE SELEÇÃO COMPOSTO}

\subsection{Métricas Base Calculadas}

Para cada ativo $i$ no período 2014-2017, foram calculadas:

\textbf{1. Momentum 12-1 (Jegadeesh \& Titman, 1993):}
\begin{equation}
\text{Momentum}_{i} = \prod_{t=-12}^{-2}(1 + r_{i,t}) - 1
\end{equation}

\textbf{2. Volatilidade Anualizada:}
\begin{equation}
\sigma_{i} = \sqrt{12} \times \text{std}(r_{i,t})
\end{equation}

\textbf{3. Maximum Drawdown:}
\begin{equation}
\text{MaxDD}_{i} = \max_{t} \left( \frac{\text{Pico}_{t} - \text{Vale}_{t}}{\text{Pico}_{t}} \right)
\end{equation}

\textbf{4. Downside Deviation (Sortino \& van der Meer, 1991):}
\begin{equation}
\text{DD}_{i} = \sqrt{\frac{1}{N} \sum_{r_{i,t} < 0} r_{i,t}^2}
\end{equation}

\subsection{Normalização em Percentis}

Cada métrica foi convertida em percentil (rank percentil) para normalização:

\begin{align}
\text{Momentum}_{rank,i} &= \text{Percentil}(\text{Momentum}_{i}) \\
\text{Vol}_{rank,i} &= 1 - \text{Percentil}(\sigma_{i}) \\
\text{DD}_{rank,i} &= \text{Percentil}(\text{MaxDD}_{i}) \\
\text{Down}_{rank,i} &= 1 - \text{Percentil}(\text{DD}_{i})
\end{align}

\subsection{Score Final Composto}

O score final foi calculado com pesos baseados na literatura:

\begin{equation}
\text{Score}_{i} = 0.35 \times \text{Momentum}_{rank,i} + 0.25 \times \text{Vol}_{rank,i} + 0.20 \times \text{DD}_{rank,i} + 0.20 \times \text{Down}_{rank,i}
\end{equation}

\textbf{Justificativa dos Pesos:}
\begin{itemize}
    \item 35\% Momentum: Maior poder preditivo (Jegadeesh \& Titman, 1993)
    \item 25\% Volatilidade: Estabilidade (Fama \& French, 1992)
    \item 20\% Max Drawdown: Controle de risco de cauda
    \item 20\% Downside Risk: Risco assimétrico (Sortino, 1991)
\end{itemize}

\section{MATRIZ DE RETORNOS MENSAIS 2018-2019}

\subsection{Metodologia de Cálculo}

Os retornos mensais foram calculados usando log-retornos:

\begin{equation}
r_{i,t} = \ln(P_{i,t}) - \ln(P_{i,t-1})
\end{equation}

onde $P_{i,t}$ são os preços ajustados da Economática (incluindo dividendos reinvestidos).

\subsection{Estatísticas Descritivas dos Retornos}

\begin{table}[H]
\centering
\caption{Estatísticas dos Retornos Mensais (2018-2019)}
\begin{tabular}{|l|c|c|c|c|c|}
\hline
\textbf{Ativo} & \textbf{Média} & \textbf{Desvio} & \textbf{Min} & \textbf{Max} & \textbf{Sharpe} \\
\hline
LREN3 & 32.2\% & 24.1\% & -9.8\% & 21.4\% & 1.07 \\
WEGE3 & 36.4\% & 25.8\% & -7.8\% & 17.0\% & 1.17 \\
ABEV3 & 0.5\% & 27.0\% & -15.7\% & 15.0\% & -0.21 \\
AZZA3 & 14.1\% & 28.2\% & -20.3\% & 18.3\% & 0.28 \\
ALPA4 & 51.2\% & 34.4\% & -14.7\% & 23.8\% & 1.31 \\
RENT3 & 46.8\% & 29.4\% & -11.0\% & 23.5\% & 1.38 \\
ITUB4 & 24.3\% & 27.9\% & -15.9\% & 23.3\% & 0.65 \\
ALUP11 & 26.2\% & 20.3\% & -10.5\% & 18.3\% & 0.99 \\
B3SA3 & 39.4\% & 29.9\% & -16.1\% & 17.5\% & 1.11 \\
BBDC4 & 27.4\% & 34.0\% & -15.8\% & 22.3\% & 0.62 \\
\hline
\end{tabular}
\end{table}

\textbf{Observações:}
\begin{itemize}
    \item Período: 24 meses (Janeiro 2018 - Dezembro 2019)
    \item Taxa livre de risco: 6.24\% a.a. (CDI médio do período)
    \item Todos os dados ajustados para eventos corporativos
    \item Nenhum dado faltante no período de teste
\end{itemize}

\section{ESTIMAÇÃO DA MATRIZ DE COVARIÂNCIA}

\subsection{Método de Estimação}

A matriz de covariância foi estimada usando a covariância amostral dos retornos mensais:

\begin{equation}
\hat{\Sigma}_{ij} = \frac{1}{T-1} \sum_{t=1}^{T} (r_{i,t} - \bar{r}_i)(r_{j,t} - \bar{r}_j)
\end{equation}

onde $T = 24$ é o número de observações mensais.

\subsection{Matriz de Correlação Entre Ativos}

\begin{table}[H]
\centering
\caption{Matriz de Correlação dos Retornos Mensais}
\scriptsize
\begin{tabular}{|l|c|c|c|c|c|c|c|c|c|c|}
\hline
\textbf{Ativo} & \textbf{LREN3} & \textbf{WEGE3} & \textbf{ABEV3} & \textbf{AZZA3} & \textbf{ALPA4} & \textbf{RENT3} & \textbf{ITUB4} & \textbf{ALUP11} & \textbf{B3SA3} & \textbf{BBDC4} \\
\hline
\textbf{LREN3} & 1.00 & - & - & - & - & - & - & - & - & - \\
\textbf{WEGE3} & 0.08 & 1.00 & - & - & - & - & - & - & - & - \\
\textbf{ABEV3} & 0.12 & 0.57 & 1.00 & - & - & - & - & - & - & - \\
\textbf{AZZA3} & 0.57 & 0.12 & -0.04 & 1.00 & - & - & - & - & - & - \\
\textbf{ALPA4} & 0.63 & 0.20 & 0.14 & 0.29 & 1.00 & - & - & - & - & - \\
\textbf{RENT3} & 0.80 & 0.02 & 0.33 & 0.37 & 0.52 & 1.00 & - & - & - & - \\
\textbf{ITUB4} & 0.60 & 0.15 & 0.39 & 0.47 & 0.36 & 0.56 & 1.00 & - & - & - \\
\textbf{ALUP11} & 0.65 & 0.39 & 0.52 & 0.35 & 0.60 & 0.63 & 0.57 & 1.00 & - & - \\
\textbf{B3SA3} & 0.56 & 0.07 & 0.39 & 0.44 & 0.46 & 0.55 & 0.68 & 0.59 & 1.00 & - \\
\textbf{BBDC4} & 0.68 & 0.15 & 0.32 & 0.50 & 0.41 & 0.58 & 0.94 & 0.66 & 0.66 & 1.00 \\
\hline
\end{tabular}
\end{table}

\textbf{Estatísticas das Correlações:}
\begin{itemize}
    \item Correlação média: 0.435
    \item Correlação mínima: -0.039
    \item Correlação máxima: 0.944
    \item Desvio-padrão: 0.220
\end{itemize}

\section{IMPLEMENTAÇÃO DA OTIMIZAÇÃO DE MARKOWITZ}

\subsection{Formulação do Problema}

A otimização de Markowitz foi implementada como problema de mínima variância:

\begin{align}
\min_{w} \quad & w^T \Sigma w \\
\text{sujeito a:} \quad & \sum_{i=1}^{N} w_i = 1 \\
& w_i \geq 0 \quad \forall i
\end{align}

\subsection{Algoritmo de Otimização}

\textbf{Método:} SLSQP (Sequential Least Squares Programming)

\textbf{Implementação:}
\begin{enumerate}
    \item Inicialização: $w^{(0)} = (1/N, 1/N, ..., 1/N)$
    \item Função objetivo: $f(w) = w^T \Sigma w$
    \item Restrição de igualdade: $\sum w_i - 1 = 0$
    \item Restrições de desigualdade: $w_i \geq 0$
    \item Tolerância de convergência: $10^{-9}$
\end{enumerate}

\subsection{Pesos Ótimos Resultantes}

\begin{table}[H]
\centering
\caption{Pesos da Estratégia Mean-Variance}
\begin{tabular}{|l|c|}
\hline
\textbf{Ativo} & \textbf{Peso (\%)} \\
\hline
LREN3 & 0.0\% \\
WEGE3 & 40.0\% \\
ABEV3 & 0.0\% \\
AZZA3 & 0.0\% \\
ALPA4 & 15.1\% \\
RENT3 & 33.7\% \\
ITUB4 & 0.0\% \\
ALUP11 & 0.0\% \\
B3SA3 & 11.2\% \\
BBDC4 & 0.0\% \\
\hline
\textbf{Total} & 100.0\% \\
\hline
\end{tabular}
\end{table}

\textbf{Verificação da Solução:}
\begin{itemize}
    \item Soma dos pesos: 1.000000 (deve ser 1.0)
    \item Todos os pesos ≥ 0: True
    \item Algoritmo convergiu: Sim
\end{itemize}

\section{IMPLEMENTAÇÃO DO EQUAL RISK CONTRIBUTION}

\subsection{Definição da Contribuição de Risco}

A contribuição de risco do ativo $i$ é definida como:

\begin{equation}
RC_i = w_i \times \frac{\partial \sigma_p}{\partial w_i} = w_i \times \frac{(\Sigma w)_i}{\sigma_p}
\end{equation}

onde $\sigma_p = \sqrt{w^T \Sigma w}$ é a volatilidade total da carteira.

\subsection{Objetivo da Otimização}

Encontrar pesos $w$ tais que:

\begin{equation}
RC_i = \frac{\sigma_p}{N} \quad \forall i \in \{1, 2, ..., N\}
\end{equation}

\subsection{Algoritmo Iterativo}

\textbf{Método:} Otimização por minimização da diferença entre contribuições

\begin{enumerate}
    \item Inicializar: $w^{(0)} = (1/N, 1/N, ..., 1/N)$
    \item Para cada iteração $k$:
    \begin{enumerate}
        \item Calcular $\sigma_p^{(k)} = \sqrt{w^{(k)T} \Sigma w^{(k)}}$
        \item Calcular $RC_i^{(k)} = w_i^{(k)} \times \frac{(\Sigma w^{(k)})_i}{\sigma_p^{(k)}}$
        \item Minimizar: $\sum_{i=1}^{N} \left(RC_i^{(k)} - \frac{\sigma_p^{(k)}}{N}\right)^2$
    \end{enumerate}
    \item Convergir quando $\max_i |RC_i^{(k)} - \sigma_p^{(k)}/N| < 10^{-6}$
\end{enumerate}

\subsection{Resultados da Otimização ERC}

\begin{table}[H]
\centering
\caption{Pesos e Contribuições de Risco - ERC}
\begin{tabular}{|l|c|c|}
\hline
\textbf{Ativo} & \textbf{Peso (\%)} & \textbf{Contribuição Risk (\%)} \\
\hline
LREN3 & 9.9\% & 10.0\% \\
WEGE3 & 15.3\% & 10.0\% \\
ABEV3 & 12.0\% & 10.0\% \\
AZZA3 & 11.4\% & 10.0\% \\
ALPA4 & 8.3\% & 10.0\% \\
RENT3 & 8.6\% & 10.0\% \\
ITUB4 & 8.5\% & 10.0\% \\
ALUP11 & 10.9\% & 10.0\% \\
B3SA3 & 8.4\% & 10.0\% \\
BBDC4 & 6.8\% & 10.0\% \\
\hline
\end{tabular}
\end{table}

\textbf{Validação do Algoritmo:}
\begin{itemize}
    \item Desvio-padrão das contribuições: 0.00\%
    \item Contribuição target: 10.0\% por ativo
    \item Algoritmo convergiu: Sim
\end{itemize}

\section{CÁLCULO DAS MÉTRICAS DE PERFORMANCE}

\subsection{Retornos das Estratégias}

Os retornos de cada estratégia foram calculados como:

\begin{equation}
r_{p,t} = \sum_{i=1}^{N} w_{i} \times r_{i,t}
\end{equation}

\subsection{Sharpe Ratio}

\begin{equation}
\text{Sharpe} = \frac{\bar{r}_p - r_f}{\sigma_p}
\end{equation}

onde:
\begin{itemize}
    \item $\bar{r}_p$: retorno médio mensal da estratégia
    \item $r_f = 0.0052$ (6.24\% a.a. / 12 meses): taxa livre de risco mensal
    \item $\sigma_p$: desvio-padrão mensal dos retornos da estratégia
\end{itemize}

\subsection{Sortino Ratio}

\begin{equation}
\text{Sortino} = \frac{\bar{r}_p - r_f}{\sigma_{down}}
\end{equation}

onde $\sigma_{down}$ é o desvio-padrão dos retornos abaixo da taxa livre de risco.

\subsection{Maximum Drawdown}

\begin{align}
\text{Valor Acumulado}_t &= \prod_{s=1}^{t}(1 + r_{p,s}) \\
\text{Pico}_t &= \max_{s \leq t} \text{Valor Acumulado}_s \\
\text{Drawdown}_t &= \frac{\text{Valor Acumulado}_t - \text{Pico}_t}{\text{Pico}_t} \\
\text{Max Drawdown} &= \min_t \text{Drawdown}_t
\end{align}

\subsection{Valores Calculados das Métricas}

\begin{table}[H]
\centering
\caption{Métricas de Performance Detalhadas}
\begin{tabular}{|l|c|c|c|c|c|}
\hline
\textbf{Estratégia} & \textbf{Ret. Anual} & \textbf{Vol. Anual} & \textbf{Sharpe} & \textbf{Sortino} & \textbf{Max DD} \\
\hline
Equal Weight & 29.84\% & 19.72\% & 1.197 & 1.584 & -18.88\% \\
Mean-Variance Optimization & 42.45\% & 19.49\% & 1.858 & 3.487 & -14.61\% \\
Equal Risk Contribution & 28.75\% & 18.62\% & 1.209 & 1.586 & -18.19\% \\
\hline
\end{tabular}
\end{table}

\section{TESTES DE SIGNIFICÂNCIA ESTATÍSTICA}

\subsection{Teste de Jobson-Korkie (1981)}

Para comparar se diferenças entre Sharpe Ratios são estatisticamente significativas:

\begin{equation}
t = \frac{\hat{S}_1 - \hat{S}_2}{\sqrt{\text{Var}(\hat{S}_1 - \hat{S}_2)}}
\end{equation}

onde a variância da diferença é:

\begin{equation}
\text{Var}(\hat{S}_1 - \hat{S}_2) = \frac{1}{T}\left[2 - 2\rho_{12} + \frac{1}{2}(\hat{S}_1^2 + \hat{S}_2^2) - \frac{\rho_{12}}{2}(\hat{S}_1^2 + \hat{S}_2^2)\right]
\end{equation}

\subsection{Resultados dos Testes}

\begin{table}[H]
\centering
\caption{Resultados dos Testes de Jobson-Korkie}
\begin{tabular}{|l|c|c|c|c|}
\hline
\textbf{Comparação} & \textbf{Diferença SR} & \textbf{Estatística t} & \textbf{p-valor} & \textbf{Significante 5\%} \\
\hline
Mean-Variance vs vs vs Equal Weight & 0.1909 & 1.9113 & 0.0685 & Não \\
Mean-Variance vs vs vs Risk Parity & 0.1875 & 2.1518 & 0.0421 & Sim \\
Equal Weight vs vs vs Risk Parity & -0.0034 & -0.1386 & 0.8910 & Não \\
\hline
\end{tabular}
\end{table}

\textbf{Interpretação dos Resultados:}
\begin{itemize}
    \item Mean-Variance vs Risk Parity: Significativo (p = 0.0421)
    \item Mean-Variance vs Equal Weight: Marginalmente significativo (p = 0.0685)
    \item Equal Weight vs Risk Parity: Não significativo (p = 0.8910)
\end{itemize}

\section{VALIDAÇÃO E REPRODUTIBILIDADE}

\subsection{Controle de Qualidade}

Todos os cálculos apresentados neste apêndice foram:

\begin{itemize}
    \item Implementados em Python 3.13 com bibliotecas científicas (NumPy, Pandas, SciPy)
    \item Validados através de verificações cruzadas e testes de consistência
    \item Documentados com código-fonte completo e transparente
    \item Baseados exclusivamente em dados da Economática (fonte oficial)
    \item Submetidos a controles rigorosos de viés temporal (out-of-sample)
\end{itemize}

\subsection{Reprodutibilidade}

Para garantir reprodutibilidade completa:

\begin{enumerate}
    \item Todos os dados utilizados são da base Economática (fonte pública e auditável)
    \item Algoritmos implementados seguem literatura acadêmica padrão
    \item Parâmetros utilizados são explicitamente documentados
    \item Seed para números aleatórios: não aplicável (sem simulação Monte Carlo)
    \item Versões de software documentadas
\end{enumerate}

\subsection{Declaração de Integridade Científica}

Declaro que todos os cálculos, dados e resultados apresentados neste estudo são:
\begin{itemize}
    \item Baseados em dados reais e verificáveis
    \item Calculados com metodologia científica rigorosa
    \item Livres de manipulação ou seleção tendenciosa
    \item Apresentados com transparência total
    \item Passíveis de validação independente
\end{itemize}

\textbf{Data de geração deste apêndice:} 09/09/2025 às 13:26

\textbf{Autor:} Bruno Gasparoni Ballerini

